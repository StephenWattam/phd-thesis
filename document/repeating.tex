The focus of this chapter is on methods for describing a sampling policy so that corpora can be repaired, augmented or resampled at a later date using methods such as weighted bootstrapping.

\section{Describing a Sampling Policy}
This will focus on two main aspects of description, but both will lean towards being algorithmically described for use in software:
\begin{itemize}
	\item Non-URI methods for describing documents online (i.e. methods including linguistic descriptions);
	\item Methods for describing, in general, a sample to an algorithm (presumably these exist already, but I'm unaware of them);
	\item Methods of evaluation ("distance between samples in terms of samples").
\end{itemize}



\section{Dissemination and Re-use of Web Corpora}
Using methods from above, the idea here is to describe how to best disseminate web corpora in a legal, reuable and scientifically valid form.  The introduction will include overviews of how people currently do it, before launching into the subsections that cover problem and proposed solutions.

\subsection{Rebuilding a Corpus}
Outline why we'd want to rebuild a web corpus in order to promote its linguistic coverage and representivity.  Produce, detail and evaluate either a method, or a piece of software based on that method, to augment or repair a corpus (software would tie in nicely to the sampler higher up).


\subsection{Reweighting a Corpus}
Explain how the size of web corpora means reweighting and resampling are expecially pertinent approaches; produce, detail and evaluate a method or software tool (as above).  Note that, algorithmically, this task is not so different to the above, so might be merged into it.


\section{Evaluation of Dissemination Methods}
Evaluate the capabilities of methods for rebuilding and resampling with some test data (again, frame with linguistic content).  Special care must be taken to evaluate similarity in a way that doesn't simply extract one or two dimensions from text, hence the subection discussing this at length.  Perhaps the subsection belongs above this section as a pre-discussion of evaluation methods.


\subsection{Similarity Measures}
Discuss the problem of evaluating a sample that has a gazillion dimensions, outline why a linguistic approach was chosen (and which one[s]).
