The focus of this chapter is on methods for describing a sampling policy so that corpora can be repaired, augmented or resampled at a later date using methods such as weighted bootstrapping.

\section{Describing a Sampling Policy}
\todo[color=blue!20, noline]{WHEN}
This will focus on two main aspects of description, but both will lean towards being algorithmically described for use in software:
\begin{itemize}
	\item Non-URI methods for describing documents online (i.e. methods including linguistic descriptions);
	\item Methods for describing, in general, a sample to an algorithm (presumably these exist already, but I'm unaware of them);
	\item Methods of evaluation ("distance between samples in terms of samples").
\end{itemize}



\section{Dissemination and Re-use of Web Corpora}
Using methods from above, the idea here is to describe how to best disseminate web corpora in a legal, reuable and scientifically valid form.  The introduction will include overviews of how people currently do it, before launching into the subsections that cover problem and proposed solutions.

\subsection{Rebuilding a Corpus}
\todo[color=blue!20, noline]{WHEN}
Outline why we'd want to rebuild a web corpus in order to promote its linguistic coverage and representivity.  Produce, detail and evaluate either a method, or a piece of software based on that method, to augment or repair a corpus (software would tie in nicely to the sampler higher up).


\subsection{Reweighting a Corpus}
\todo[color=blue!20, noline]{WHEN}
Explain how the size of web corpora means reweighting and resampling are expecially pertinent approaches; produce, detail and evaluate a method or software tool (as above).  Note that, algorithmically, this task is not so different to the above, so might be merged into it.


\section{Evaluation of Dissemination Methods}
\todo[color=blue!20, noline]{WHEN}
Evaluate the capabilities of methods for rebuilding and resampling with some test data (again, frame with linguistic content).  Special care must be taken to evaluate similarity in a way that doesn't simply extract one or two dimensions from text, hence the subection discussing this at length.  Perhaps the subsection belongs above this section as a pre-discussion of evaluation methods.


% 
% % Moving from problem to solution again
% \section{Tools Based on the Chosen Methods}
% A major contribution here is to be a suite of software to allow people to put some of these principles into practice.  The idea is to use this as a separate paper, and merge it in here, so the subsections are quite fluid.
% 
% \subsection{Details of Implementation}
% Outline the design choices made to avoid each of the issues.  There'll be a focus on having some kind of sampling strategy descriptor because it ties nicely to the next chapter.
% 
% \subsection{Evaluation}
% Here  I'll attempt to evaluate the quality of sampling.  This will be done by comparison of results from linguistic analyses, such as keywording, as well as in comparison to the findings of search-engine document attrition studies (these will validate the findings to some extent, rather than merely stating that the corpus is different).
% 
% The big question here is how I evaluate corpus quality without using a larger (ideally more biased) corpus.  Chances are this method will become a subsection of its very own.
% 


\subsection{Similarity Measures}
\todo[color=blue!20, noline]{WHEN}
Discuss the problem of evaluating a sample that has a gazillion dimensions, outline why a linguistic approach was chosen (and which one[s]).
