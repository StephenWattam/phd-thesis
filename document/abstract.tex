% TODO 



%The process of corpus building has traditionally presented many practical problems, from issues surrounding licensing to considerations of how to digitise and collect data.  Subsequently, the quality of data in corpus linguistics varies wildly around these issuyes

\todo[color=blue!20, noline]{Redo}
{\bf TODO: rephase abstract in line with revised contents}

Corpus linguistics has long claimed validity through its claim to representative samples of language --- indeed, some go so far as to advocate basing all findings entirely on bottom-up inference about these samples.  The basis for claiming a representative sample, however, is often misrepresented, misunderstood, or simply unaddressed.

Over time, however, the approach used to sample many widely-used corpora has proven itself practical and applicable to many useful tasks.  Even so, studies often mention the ill-fitting nature of their corpora, and it is reasonable to presume that there is still significant room for improvement, both for the testing of specialised, domain-specific methods and the construction of wider, more standardised, general purpose corpora.

\paragraph{}
Through time, corpora have inspired a number of causal theories as to the nature of language use --- I argue that it is these theories, chosen in an informed manner for each task, that should underpin the use of corpora for research.  Application of them to the process of sampling and balancing corpora provides a theoretically defensible base around which the findings of a given study, or comparisons to another, may be based.

\paragraph{}
As new technology emerges that expands our ability to sample new areas of language use (such as the web, speech recognition or portable devices), we are provided with the tools to correct many of the working assumptions made when building the first general purpose corpora.  The methods to provide this are covered in section [x]

\paragraph{}
These new areas of language use also provide unique challenges of their own --- we focus on one such area, the web, in order to quantify the effect time has on document availability (and hence the linguistic content contained within).  We go on to devise methods for correcting the most damanging of biases in this area, and contribute software tools for others to use.

%-- possibly?
\paragraph{}
Through application of techniques from (c) to existing general-purpose corpora, we present a method for [rebalancing if possible, acquiring otherwise] language in order to better satisfy the underlying theoretical assumptions of language.  It is our hope that this may lead to a step improvement in the accuracy and/or generality of techniques derived from quantitative analysis of test.


