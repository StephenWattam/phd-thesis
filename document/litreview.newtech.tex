

\til{New technologies have "fixed" many issues surrounding the availability of data, but introduced their own intricacies.  Things such as text to speech, life corpora and the web will be introduced here, after which the web will be focused on.  The intent is to frame web corpora in terms of their emerging nature w.r.t. other sources of data, and to state that we wish to exceed, rather than emulate, the quality of existing corpora.}


\subsection{Text Sources}
\subsubsection{Web Corpora}
\til{This section will cover, in general, web-related concerns.  The aim is to provide a comprehensive overview of the challenges to good scientific practice and representative samples.

I envisage this section having a subsection for each issue, for example, attrition, cleaning of data, identification of documents, etc.}

\subsubsection{Life-Logging}
Cover the possibility of using low-impact audio recording or web proxies for corpus collection.

\subsubsection{Digital Documents}
Documents that start off and remain digital.


% ---------------------------------------------------------------------------------------------------------------------
\subsection{Sampling Methods}

\subsubsection{Access}
Explain how this greatly improves methods of sampling through access to people.  Systems such as AMT are a boon especially to special purpose corpus builders.

\subsubsection{Indexing and Retrieval}
Systems such as search engines (private ones or google), data warehousing and noSQL all offer advantages in retrieval that may be used to resample and retrieve data from existing corpora in an efficient and practical manner.

%CITATION: "Googleology is bad science", kilgarriff
