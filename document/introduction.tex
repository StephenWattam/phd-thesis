%% sample.
%\begin{itemizeTitle}
%\item[Title of my first item] Text of my 1st item.
%\item[Second one] And some text here.
%\end{itemizeTitle}
%
%I spent three years drinking beer 
%and didn't have any time to write a thesis.  
%But I deserve a PhD anyways.

% TODO

% About corpora and corpus linguistics
% 

%To say that corpus linguistics has come of age would be an overstatement.  Corpus methods have weathered many storms regarding the philosophical basis of findings, and have been able to defend themselves with useful and convincing results, however, there remains little formal justification for much of the work, 
   
The aim of corpus methods in linguistics is, basically, to ground inference in empirical evidence.  This is a significant improvement upon alternative methods, but is abrogated in no small part by the difficulty in constructing corpora that are expansive and representative enough to cover meaningful portion of language whilst remaining free of significant biases applied by those tasked with building them.

Many corpora, though still significant improvements upon non-corpus methods, use methods for selection and construction that echo past efforts, without building upon them in ways that would ensure statistical improvement.  In part this has been due to the scientific pressure of being able to compare ones' results, forming a body of `comparable' corpora each using the same sampling frame.

The rise of \textsl{web as corpus}, and more recently \textsl{big data}, has provided us with the tools needed to build corpora with very little supervision.  This allows us to re-examine and, in places, exceed the limitations of conventional corpora.  Even so, many corpora are compared against conventional ones due to their status as a de-facto gold standard of representativeness.

In this thesis I'll provide justification for the development of new sampling procedures, guided less by concepts of linguistic balance, intuition and coverage, and more by statistical sampling theory.  I apply these methods to the problem of improving WaC samples (whilst avoiding their novel pitfalls), and present ways in which the whole scientific process of corpus comparison, dissemination and replicability may be assessed.
