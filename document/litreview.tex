
\section{Sampling in Corpus Linguistics}
The review of literature begins with a discussion of the current state of affairs w.r.t. sampling in corpus linguistics: how do people go about acquiring corpora, what their motivation and rationale is, and what people are currently developing in terms of sampling methodology.  I'll attempt to include some historical rationale too here.

\subsection{Criticism of Current Data}
Here I will go on to discuss the failings of current widely-used lexical resources from the standpoint of their common use --- the section will focus on the representativeness of studies based on corpora as a method of commenting on their quality, and it's therefore necessary to draw the distinction between:
\begin{itemize}
	\item General purpose corpora, and;
	\item Specialist corpora (and therefore some methods people use to build them).
\end{itemize}



\subsection{Current Discussions of Representativeness}
Here I'll focus on work that is in the same vein as the thesis, i.e. how can we go about improving things?



% Here there is a conceptual shift from "Current practice//problem" to "Ideal practice//solution"
\section{Formally Sampling Language}
This will be a review of more formal sampling theory, comparing it to methods for acquiring language.

\subsection{Theoretical Bases for Sampling Language}
The aim in this section is to draw out prominent linguistic concepts and justify sampling methods examined above in terms of their suitability to the theory of how we use language.  This is the tie to the personal corpus stuff, but will mention other ideas of what language is.

\subsection{Comparison of Methods}
Here I'll draw parallels between sampling methods and comment critically on their value.



% Now we shift onto the low-level focus of the thesis, web issues
\section{New Technologies} % "Opportunities and Challenges"
New technologies have "fixed" many issues surrounding the availability of data, but introduced their own intricacies.  Things such as text to speech, life corpora and the web will be introduced here, after which the web will be focused on.  The intent is to frame web corpora in terms of their emerging nature w.r.t. other sources of data, and to state that we wish to exceed, rather than emulate, the quality of existing corpora.


\subsection{Web Corpora --- Challenges}
This section will cover, in general, web-related concerns.  The aim is to provide a comprehensive overview of the challenges to good scientific practice and representative samples.

I envisage this section having a subsection for each issue, for example, attrition, cleaning of data, identification of documents, etc.

