
\section{Sampling in Corpus Linguistics}
\label{sec:sampling-corpus-linguistics}
\til{The review of literature begins with a discussion of the current state of affairs w.r.t. sampling in corpus linguistics: how do people go about acquiring corpora, what their motivation and rationale is, and what people are currently developing in terms of sampling methodology.  I'll attempt to include some historical rationale too here.}

\subsection{Criticism of Current Data}
\til{Here I will go on to discuss the failings of current widely-used lexical resources from the standpoint of their common use --- the section will focus on the representativeness of studies based on corpora as a method of commenting on their quality, and it's therefore necessary to draw the distinction between}
\begin{itemize}
	\item General purpose corpora, and;
	\item Specialist corpora (and therefore some methods people use to build them).
\end{itemize}



\subsection{Current Discussions of Representativeness}
\til{Here I'll focus on work that is in the same vein as the thesis, i.e. how can we go about improving things?}



% Here there is a conceptual shift from "Current practice//problem" to "Ideal practice//solution"
\section{Formally Sampling Language}
% This will be a review of more formal sampling theory, comparing it to methods for acquiring language.
As we have seen in section~\ref{sec:sampling-corpus-linguistics}, corpus building efforts generally seek to solve a number of practical problems with quantitative linguistics, namely:
\begin{enumerate}
    \item Defining a finite and reliable linguistic resource (one that is static and entirely accessible);
    \item Replicating, comparing, and disseminating results.
\end{enumerate}

As such, the focus of many corpus building efforts \td{which corpus building efforts?}
has been largely on problems of procuring texts, selecting sufficiently broad categories of texts to be useful to many researchers, and managing the practicalities of text selection (i.e. by taking snippets of published and copyright works rather than their whole inclusion).

The use of corpora availed many statistical quantitiative techniques, each bringing a series of assumptions regarding not only the internal nature of the corpus, but also its relation to the larger population.  Whilst the validity of internal models has been discussed at length\td{where discussed?}
, little has been done to address the problems of external validity in this regard.

Therefore, in order to improve (and to assess) the value of a corpus to the wider field, it is necessary to inspect not only its empirical advantages (now well-determined for many frequently-used corpora) but also the potential value any given corpus building stratgy may yield in the ideal case.

In this section, I will be offering a critical description of how linguistic data may be sampled using existing, conventional, statistical techniques.  This comparison shall act as a gold standard against which existing (and future) efforts may be judged, as well as offering a stance from which to assess the successes and failings of current corpus-building efforts.



\subsection{The ideal Corpus}
\label{sec:sub:ideal-corpus}
\til{Based on current usage, but notably not restricted by methods and practicalities, review 2.1 with a view to extracting what we want from a corpus---address size, randomness, validity (internal + external), replicability, reliability, and flexibility.}

\subsubsection{Nature of Sample}
\til{Conventional corpora are guided by experts.  Do we want pure random corpora?  What does this even mean, what would the population be? language use or language exposure?  Defer some of these until the later sections of the thesis.}
\subsubsection{Validity}
\til{Following on from the above.  What does external validity become for a general/specific purpose corpus?  Do common corpora have any claim to external validity when used for qualitative/quantitative analyses of varying types?}
\subsubsection{Replication and Reliability}
\til{Corpora are hailed as being a basis for collaboration.  This is true of the most broken corpus, but a shared and accepted broken sample is worse than everyone sampling from the same population.  Address subsampling problems and issues of 're-usable bias' for very popular corpora.  To what extent are studies based on the same corpus comparable when using massively  different methodologies?}
\subsubsection{Size}
\til{How many words should a piece of string have in it?}
\subsubsection{Flexibility}
\til{Selection of data for qualitative research, or subsampling for quantitative, has the potential to undo all of the balancing above if not done properly.  How can this risk be minimised (guidelines for researchers, corpus size, specific corpus design, distributing tools with corpora)?}
% ---
\subsubsection{Summary}
\til{A simple, short and refer-able list of important things to work for in the below section.  Really, this might section up to here might be a good paper for CL or something.}

% \subsection{Paradigms for Sampling Language} % Kuhnian
% \til{The aim in this section is to draw out prominent linguistic concepts and justify sampling methods examined above in terms of their suitability to the theory of how we use language.  This is the tie to the personal corpus stuff, but will mention other ideas of what language is.}



\subsection{Comparison of Methods}
\til{Here I'll draw parallels between sampling methods and comment critically on their value.}
\subsubsection{Population}
\til{This is a largely linguistic issue of external validity, and will point back at section 2.1 a lot}
Linguistic issues (p'raps best taken from 2.1?):
\begin{itemize}
    \item How can we select a population closest to that discussed in section~\ref{sec:sub:ideal-corpus}?
\end{itemize}
Statistical theory:
\begin{itemize}
    \item How other fields with very complex parameter spaces do sampling
    \item Methods for population estimation given constraints from section~\ref{sec:sub:ideal-corpus}
\end{itemize}
\subsubsection{Sampling Frame}
\til{This is largely an issue of internal validity and reliability.}
Enumeration or bounding of a population:
\begin{itemize}
    \item General and special purpose corpora
    \item Sources of data (web, books)
    \item Time
\end{itemize}
Issues surrounding subsampling [general purpose corpora], and how this ought to be done

Sources of valid auxiliary information for sampling
\subsubsection{Sampling Methods}
\til{An overview of statistical sampling methods relevant to ling.}
Nonrandom methods, primarily included to compare against current ones:
\begin{itemizeTitle}
    \item[Purposive] Compare against current techniques.  Critique use of inferential stats on such corpora.  Big data perhaps illustrates flaws?
    \item[Snowball Sampling] Relevant due to web crawlers
\end{itemizeTitle}
Random sampling methods, as possible approaches
\begin{itemizeTitle}
    \item[Simple Random] Ideal case.  Focus on what makes it hard for linguistics.  Attempt to identify ways around this (empirical estimates of P()). Detasil practical issues with selection relative to stats.  
    \item[Stratified] How to select strata (w/ling. relevance)? Multi-dimensional strata. Inverse probability weighting.  PPS possibilities with web/offline.
    \item[Multistage ~and~] IPW, how to balance samples and...
    \item[Cluster] Website-wide clusters.  Relate to the structure of data, with publishers/websites forming clusters.
    \item[Adaptive] Perhaps too nonrandom for this section?  Offers a way to balance corpora with linguistic sampling frame.  Select frame using multistage sampling?
\end{itemizeTitle}

\subsubsection{Size and Power}
\til{Methods for selecting size and power.}
Linguistic criteria for selecting sizes (relate to ling. lit above) [common models, sufficient representation of odd features, breadth of coverage].

Existing corpora's approach (refer to 2.1 mainly).  Focus on the rise of big data+simple models showing that p'raps small data is biased ("bad smells").
\subsubsection{Reweighting}
\til{Multistage sampling and bias estimating with and without using auxiliary data.}
For multistage designs, how large should the initial corpora be? Is it small enough to be driven by the user's data or selection of medoid items?

Auxiliary Data:
\begin{itemize}
    \item Sources of valid auxiliary data (without inheriting the bias of existing corpora, ideally)
    \item How broken manual resampling of corpora is at the moment, and how people *want* to do it.
\end{itemize}
\subsubsection{Bias Estimating}
\til{This covers the problem of evaluation, how do we know that a corpus is better than another without a gold standard?  Does a gold standard actually exist, but we are not using it as such?  How can we ensure change goes in the correct direction?}
Repeated resampling and benchmarking.

Comparison to auxiliary data, suitable distance metrics, and criteria for acceptance.








% Now we shift onto the low-level focus of the thesis, web issues
\section{New Technologies} % "Opportunities and Challenges"
\til{New technologies have "fixed" many issues surrounding the availability of data, but introduced their own intricacies.  Things such as text to speech, life corpora and the web will be introduced here, after which the web will be focused on.  The intent is to frame web corpora in terms of their emerging nature w.r.t. other sources of data, and to state that we wish to exceed, rather than emulate, the quality of existing corpora.}


\subsection{Web Corpora --- Challenges}
\til{This section will cover, in general, web-related concerns.  The aim is to provide a comprehensive overview of the challenges to good scientific practice and representative samples.

I envisage this section having a subsection for each issue, for example, attrition, cleaning of data, identification of documents, etc.}

