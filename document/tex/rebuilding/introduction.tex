

We have seen in previous chapters (\td{which}) a number of scientific and practical issues that affect current corpus collection methods.

This chapter details the design and implementation of a tool that is designed to aid a number of common corpus building workflows.  These workflows are designed both as a continuation of the methods detailed in chapter~\td{chapter 4} and as part of work with existing corpora, and represent areas of scientific importance in corpus studies.

The chapter begins in \td{ref} with an outline of the aims of the tool, detailing the workflows it is designed to assist with, and identifying current methods' strengths and weaknesses.

Immediately following is an outline of the design of the tool, specifying the differences to existing tools/procedures and describing the design of the tool.

The method section (\td{ref}) describes the mechanisms used to implement these designs, and covers the architecture of the software as built.  It discussed the heuristics used, as well as the variables selected for the test system.

The implementation section (\td{ref}) specifies the technical details of the solution, and provides details on the approximation and optimisation methods used.  


Finally, the conclusion relates the design of the tool back to the aims detailed in \td{ref chapter 2}.
