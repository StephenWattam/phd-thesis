The open publishing model of the web leads to a far faster turnover of documents compared to most conventional printed sources.  In addition, the lifespan of a document is controlled by its author, rather than by any publishing party, often requiring the continuous provision of funds merely to stay available.  The natural conclusion of this is that documents available online are there in a transient, time-limited sense only --- in order to uniquely identify a document online, both a URI and time are needed.

The phenomenon of documents becoming unavailable over time, known as `link rot', has been studied at length by those who seek to index the web, most notably search engine designers.  Lately work has also been done to identify flaws in the longevity of scholarly collections\cite{10.1371/journal.pone.0115253}.

Document change over time also has important ramifications for those working with web data.  Aside from changes in owernship of domain names and site restructuring, many web pages undergo gradual revision and updates, particularly those in the news genres.  Again, whilst there has been study of these events for the purposes of crawler design, the linguistic impact is less known.

As WaC methods become adopted, and other general-purpose corpora seek to include web data, the distribution of documents through time becomes an important sampling issue: the age of information when read, and the longevity of it when written, both affect the veracity and relevance of any conclusions.  As corpora age, there is also the possibility that this attrition correlates with genre, meaning that a predictable bias creeps into the dataset.

Current crawler design (and thus the contents of search engine indices) focuses on keeping the most current page at any given time.  This may be insufficient for many scientific purposes, which must know the context and contents of each page over time, and do so in a manner comparable to other documents from that time period.

This chapter starts in Section~\ref{sec:longitudinal:attrition} with an examination of current open-source corpora, and their likely half-lives online.  This is a motivating example to frame the introduction of the LWAC tool (Section~\ref{sec:longitudinal:lwac}), which provides a mechanism for longitudinal sampling of large corpora over time.  The suitability of this tool for long-term use is evaluated therein, before a summary (Section~\ref{sec:longitudinal:summary}.


