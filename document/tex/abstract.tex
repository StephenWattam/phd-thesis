Current efforts in corpus linguistics and natural language processing make heavy use of corpora---large language samples that are intended to describe a wide population of language users.

% Historically these items have had their utility abrogated by a number of economic, social and practical limitations.

The first modern corpora were manually constructed, transcribed from published texts and other non-digital sources into a machine-readable format.  In part due to their hard-won nature, larger corpora have become widely shared and re-used: this has significant benefits for the scientific community, yet also leads to a stagnation of sampling methods and designs.

The rise of Web as Corpus (WaC), and the use of computers to author documents, has provided us with the tools needed to build corpora automatically, or with little supervision.  
This offers an avenue to re-examine and, in places, exceed the limitations of conventional corpus sampling methods.  
Even so, many corpora are compared against conventional ones due to their status as a de-facto gold standard of representativeness.  Such practices place undue trust in aging sample designs and the expert opinion therein.

%---

In this thesis I argue for the development of new sampling procedures guided less by concepts of linguistic balance and more by statistical sampling theory.
This is done by presenting three different areas of potential study, along with exploratory results and publicly-available tools and methods that allow for further investigation.

The first of these is an examination of temporal bias in sampling online.  I present a preliminary investigation demonstrating the prevalence of such effects, before describing a tool designed to reveal linguistic change over time at a resolution not possible with current software.

Secondly, the sample design of larger general-purpose corpora is inverted in order to relate it to an individual's experience of language.  This takes the form of a census sample of language for a single subject, taken using semi-automated methods, and illustrates how poorly suited some aspects of general-purpose corpora are for questions about individual language use.

Finally, a method is presented and evaluated that is able to describe arbitrary sample designs in quantitiative terms, and use this description to automatically construct, augment, or repair corpora using the web.  This method uses bootstrapping to apply current sampling theory to linguistic research questions, in order to better align the scientific notion of representativeness with the process of retrieving data.

%I apply these novel methods to the problem of improving WaC samples (whilst avoiding their novel pitfalls), and present ways in which the whole scientific process of corpus comparison, dissemination and replicability may be assessed---itself something that demands an alternative view of corpus goals.I go on to present tools that make these tasks easy to apply in a scientific context, in order to construct usable and practical web-derived corpora.


