Large samples of text, known as corpora, are important for many users.  The groups I will be focusing on for the purposes of this thesis are particularly interested in scientific samples, that is, in generalising findings from the corpus to a wider population.  This includes much work in Natural Language Processing (NLP) and corpus linguistics, and some of what is commonly termed `data science'\footnote{Though in most `data science' contexts, sampling is performed as an intrinsic part of the task}.


The problem of sampling linguistic data is a well-researched one in the field of corpus linguistics, and a review of linguistics' approach to corpus construction will form the bulk of this chapter.  This portion of the review begins with a historical overview of approaches to corpus construction, before focusing on the problem of defining a corpus in sufficiently concrete terms to suit the rest of the work presented in the thesis.  It then focuses on potential threats to validity exposed by these methods, both from the point of view of corpus construction and use.

Not to be neglected, however, is the perspective offered by more formal survey sampling, especially as its use across the social sciences can offer insights into solutions for practical problems.  In Section~\ref{sec:litreview:sampling} a number of appropriate sample designs are identified and evaluated with respect to corpus sampling.

Finally, in Section~\ref{sec:litreview:newtech} the first few steps towards improving sampling methodology within corpus linguistics are reviewed, with a focus on web-based corpus construction methods.

