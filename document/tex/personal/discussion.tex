

The quantitative data presented above are of limited utility to others, since the degree to which they describe the subject's life (rather than some general linguistic trend) is unknown.  Without further sampling, or detailed linguistic auxiliary data, we cannot begin to generalise from the above in a useful manner.

This does not mean, however, that we cannot reason rationally about how transferrable those results are---it is more unlikely that a member of the general population is a software developer than it is that they watch TV, or listen to BBC Radio 4 in the mornings.

Moreover, since the purpose of the study is to assess the viability of methods, these may be seen as significantly more transferrable than the data they have collected (this distinction is blurred significantly by the inclusion of a human interest model, however.)




% ---------------------------------------------------------------------------
\subsection{Types of data}
The form each corpus takes is also rather variable due to the different sampling methods used.  For some sources, e.g.\ books, the personal corpus is likely to contain more partial samples (as books are seldom read in entirety within one `linguistic event').  Those sources that are read piecemeal are also likely to occur multiple times, something that would not be tolerated in conventional corpus designs.
% ---------------------------------------------------------------------------




\subsection{Method}
The process of gathering data itself was optimised for low intrusiveness, and largely proved practical to pursue medium-long term.

Maintenance of the on-line notebook was the major interruption in everyday language use, and this was gradually optimised to a shorthand format that demanded less time in the field, yet more annotation to extract data.  Where the subject is someone other than the analyst, the format of journals would need to be solidified ahead of time to ensure all properties are identifiable.

On reflection, the mnemonic effects of the notebook were most useful in operationalising the data---the narrative they created placed all other data items in context, and the detailed notes in the nightly journal contribute to an episodic memory that aided the coding process.  Again, this process of reflection is unlikely to be made available to an analyst without explicit insertion of an interview phase where the two may meet to resolve possible misconceptions.

The nightly journal, whilst useful during the study as a place to write down easily-forgotten details of language events, was less useful than anticipated during analysis.  In reality, the richness of the narrative one quickly notes down of a night proved to be difficult to operationalise, meaning that it was largely of use only as auxiliary data to augment the on-line notebook's event-by-event format.

Some sources of data proved significantly easier to normalise into a workable format than did others.  (The suitability of each will depend heavily on the form of data one is aiming for when deciding to use similar methods, however).

SQUID logs proved particularly difficult to process largely due to technical reasons, as extensive whitelists and multiple manual inspection phases were necessary to disregard advertising and AJAX requests.  A possible solution to this would be to rely on sampling of web data closer to where it is consumed, for example through the use of a browser plugin.

Initial plans were to use Voice Activity Detection (or full automated transcription) to estimate word counts from verbatim audio recordings.  In practice the diversity of unwanted noise effects (background noise, positioning of microphone, overheard conversations) rendered this practically impossible.  Even rudimentary VAD algorithms work with frequency detection strategies, and many were liable to detect things such as music as continuous speech.  %\td{say more on this}



% ---
Of particular interest to the methodology is the inclusion of a human interest model in the processing stages.  Without this model, the data is changed to massively overestimate the word counts of many data sources (some more than others), and it is the opinion of the subject that the model improves the plausibility of data.  Generalising the method requires generalising this model or narrowing down the number of sources it must cover (either by using more selective data gathering methods or by restricting the scope of the sample).  

Each of the data sources mentioned is burdened with its own empirical concerns that must be considered when designing a human interest model, and the general solutions for many of these may be particularly complex.  For example, models for websites using large graphical elements as well as text must take into account Gestalt principles as well as models for humans as they read text, and the particular interests of the subject.

One way this problem may be solved is through detailed elucidation of particular features and interests via a questionnaire or laboratory tasks in addition to the data gathering methods described here.  Clearly, though, relating these to a usefully-complex model of human interest will demand significant further research.




One of the larger challenges in gathering data from so many sources (and in so many forms) is the amount of work needed to operationalise it.  This was done both manually (in the case of complex data such as photographs and notebook entries) and automatically (for the majority of sources).  Often, some degree of manual correction or annotation was necessary even where automated processing was used.  These processing tools were bespoke, and would need to be generalised if the method is to become viable for use gathering further samples.



% ---
A number of questions were raised during the sampling period surrounding the limits of data that should be captured.  The solutions used were taken from the subject's own judgement of what constituted language `use' (since this case study is predicated upon recording that opinion), however, further work would be needed to establish answers to these in the general case, and other researchers may be guided by more principles linguistic theory.


The attention models used to narrow down input have already been mentioned, however, they apply at a very specific level---often, shorter texts such as signs, brands and labels would have particularly familiar or conventional text placements, styles, and shapes (some media, such as road signs, deliberately accentuate these features to aid recognition).  It is unclear at what point a text is `read' rather than just recognised in the periphery of one's vision.  The solution used in this study was one of internal vocalisation---if the text was recognised sufficient to repeat/act upon it mentally, it was classed as read.  This means it is often possible to say that a text source had just a few words read, when in fact a number of boilerplate features were skipped over because of their position and style.


An extension of this problem is one of re-reading texts that are being written.  The degree to which this occurs is likely extremely variable by person and task, however, it proves to be a particularly complex form of the above problem and is particularly hard to measure (or even subjectively assess).  In this case study no attempt was made to compensate for this effect.  Detailed study using some kind of attention measuring system (for example, eye tracking) may allow for deliberately using attention coefficients greater than 1 (for word counts) or deliberate repetition of data in the final corpus.


The suitability of the attention system used in this case study is also debatable.  The current method of using a coefficient works only for estimating word counts, and does not favour certain regions of the final data over others.  Additionally, it conflates two effects---that of reading only a small proportion of the source text, and that of paying little attention to the text (you may also consider using a second text at the same time a third form of inattention).  Annotation of simultaneous events was fairly simple in the notebooks, however, estimation of the distribution of one's attention was done only at a low resolution (the codes used practically equate to ratios of 80/20, 50/50, 20/80, 100 with only a few exceptions).





Since the aim of this study was to assess genre proportions, significant amounts of effort were saved by not converting all sources into verbatim text.  This is largely an issue of post-processing, as many methods were digital and thus yielded verbatim text with ease and those that do not have well established transcription procedures.  Of particular note is the importance of being able to apply a regional attention model to extract (or weight) the most read portions of a text, something that (as noted above) was not attempted here due to a focus on proportions and word counts.






\subsection{Sampling Period \& Validity}
It seems clear both from personal reflection and examination of the data, which strongly favours certain work-related activities over others, that a two-week sampling period is hardly enough to represent even an individual's language use.

There are a number of obvious reasons for this, particularly egregious examples being:

\begin{itemize}
    \item Some language sources are usually read slowly, such as books read at bedtime.  A sampling period that covers an individual's various literary interests would have to be very long indeed.
    \item Life is strongly periodic, and though attempts were made to cover the weekly cycle of work and weekend, many events occur annually (either due to seasons or social convention).  There are no Christmas songs in my corpus.
    \item Many more obscure items were represented only once in the corpus (such as advertising on vehicles or documentation on things such as legal forms).  These features may not be periodic but they are temporally rare.
    \item A person's behaviour is likely to be `chunky', following one pattern for a time and then deviating suddenly (for example, during holidays or upon a job change).  Significant effort would have to be given to careful description of duration and circumstance in order to make confident generalisation from the data captured using this method.
\end{itemize}

It is my opinion that this method is up against two major challenges which must be balanced in order to achieve some degree of scientific validity.  The former, and most addressable, of these is the difficulty of sampling in-the-field.  The more obtrusive a method, the higher quality data is going to be, however, the shorter any practical sampling period is likely to be.  The latter, as already discussed, is the problem of generalising between people in order to create a corpus of inter-person language use that retains the same details discussed here.

Given the relative paucity of metadata within many general-purpose corpora and the periodic/temporal complexity of life, it seems reasonable to suggest that an ideal mix would be better formed from long-term (multi-year) samples using low-resolution sampling techniques.  This may entail, for example, discarding the on-line journals and manual photography in favour of automated life-log photography and a nightly journal only.

Following sufficient repetition of short-duration studies, it would be possible to be more confident in the results of longer-term, lower-resolution samples, as the areas with less uncertainty may be attended to less intrusively.







\subsection{Data and Comparisons}
As mentioned in the quantitative section above, there are some large differences between the corpus gathered here and the BNC (here used as an example general-purpose corpus that purports to cover the subject).  It is clear that a number of these are down to individual variation (the technical and musical genre bias), however, other disparities are of an altogether more ambiguous status.

The overall word count of the BNC, and other similar corpora, is called into question by the size of this corpus.  In two weeks, a single person used almost a million words whilst focusing heavily on just a few subjects and roles.  Given the sheer number of people within the British population, their demographic variability, and the dubious extent to which even these two weeks represents our subject's use of language accurately, it seems preposterous to suggest that a mere 100 million (or even billion) words would sufficiently represent language use for almost any purpose.


As previously discussed, the different selection of genres indicates that the BNC's selection of materials is an imperfect superset of the subject's use of language.  It is reassuring that many of the genres absent from the BNC are minor ephemera (signs, tax discs etc.), however, there is a significant underrepresentation of others (mainly technical and broadcast media).  This underrepresentation is partially due to the subject's demographics, something that we would expect to skew the relative proportions of the corpus, and partially due to the age of the BNC, something that is particularly undesirable.



% \til{probably some more...  What is the ratio of variance between a population and the individual, over a given time period?}









\subsection{Validity}
The design of this experiment is subject to a number of challenges to validity.

Perhaps the most severe of these is the extremely personal nature of the corpus itself, which renders verification of the data all but impossible except through elicitation and subjective judgement.  This is to some degree a property of all case studies, especially those seeking to experiment with methodology.

A strong case has been made in many fields (and by the inaccuracy of certain assumptions made during preliminary tests within this study) for the fallibility of subjective opinion, and this is partially the reasoning behind a census methodology---it is a simpler (and hopefully less controversial) task to mechanically and objectively record each linguistic event than it is to estimate their size.

The main subjective component of the method used here lies before the linguistic events themselves are recorded: judging when a text is read, rather than unthinkingly `seen'.  The assumption made for this case study is hopefully uncontroversial enough to be accepted for a majority of purposes: after all, it seems unlikely that we will be able to develop an objective and meaningful threshold for this.

The primary challenge to validity due to subjective reasoning occurs after the data is captured.  This is where issues that lie beyond the scope of this thesis are---development of a robust and generalisable human interest model being a major one that has been shown to make a large difference to the results.  The model used for this study is deliberately and knowingly subjective, and would not be applicable to any replication effort.

% ---
From another perspective, the data set described here is difficult to relate to existing literature due to its orthogonal sampling structure---the whole corpus represents a single (albeit very rich) data point in most other corpus designs, and this robs us of quantitative knowledge of how it relates to other data sources.

This question of generalisability has been attempted by a rational comparison with corpora of known demographic coverage above.  A better method still would be to extract only those texts from a corpus that match the demographics of the subject described here.  Unfortunately, this is not possible to any useful degree given the limited information on the users of texts within conventional corpora, and so a more detailed comparison is again stopped by the lack of known-similar data.

These limits on comparison to others are both lifted by restriction of the data types being covered, especially where those types are easy to sample.  It would be possible, for example, to build a special-purpose web history corpus in which to contextualise a user's web history, and use this to impute their position in larger corpora with greater-than-zero confidence.






\subsection{Ethics}
\label{sec:personal:discussion:ethics}


The increased resolution of data pertaining to a single individual renders the methods discussed here ethically sensitive.  This sensitivity is increased further if continuous recording of audio or video are used, though, as mentioned above, this data was not integrated into my analysis.

% There are a number of arguments justifying covert research in the social sciences, and ...

Further, future developments in the methods described may use questionnaires or other less-invasive methods as sources of auxiliary data.  These would be targeted to a particular study design and need not cover the full set of language uses, mitigating any ethical concerns by limiting the descriptive power of the raw data itself.

A number of technical measures are also possible that may assist this issue---some of these have been developed by Ellis \& Lee working on the Machine Listening project~\cite{ellis2004audiolog}, who irreversibly scramble their audio recordings in such a way that VAD algorithms may still run.  A further option is streaming of data to a remote server, which can process, summarise, and discard verbatim data on-the-fly to prevent any possible information security breaches.


Particular sources of data raise various ethical and legal issues. These are generally centred around the presumption of privacy, which applies to most direct personal interaction, as well as some more formal communications (such as email)\cite{rampton1994baal}.  This relationship is described by Herrera~\cite{herrera1999covert}:

\begin{quote}
Covert research is also not, on reflection, so much like the conversations that we casually engage in. On the contrary, covert research raises a unique problem: those closest and most vital to the ‘conversation’ are valuable only so long as they know less about it than anyone else.
% The fieldworker has an ulterior motive of which his subjects are not aware when they kindly invite him to tea or shower him with Christmas cards\td{Find the source of this quote, it seems to have vanished!}
\end{quote}

Many of the ethical objections posited in sociological texts apply to the misrepresentation of a field worker, leading to an asymmetric relationship between him and other participants, and necessitating various abrogations of personal and professional ethics. These issues are essentially of no relevance to a simple and private recording process that does not entail entering any situations contrary to the researcher’s normal behaviour.

Any use of covert methods should always be justified in terms of the importance of the research, lack of alternatives, and sensitivity of the data to be gathered. Clearly, recording audio will require a higher standard of justification upon each of these, and the counterarguments below are written with both approximate (recording genre, setting etc) and verbatim methods in mind:

\begin{itemize}
    \item Eliciting consent would disrupt linguistic content, something that is minor for large interactions but makes representation of shorter utterances and conversations impossible. As one aim of the sampling design used here is to capture a more empirical dataset, any attempts to elicit consent must be less disruptive than those used to build other corpora.
    \item Without survey of all language used, it is impossible to assess the validity of hypotheses in the study.
    \item Language data recorded is unlikely to be directly personally identifiable, or culturally/commercially/governmentally sensitive. This will depend on the status and role of the researcher, of course.
    \item Results may be reported without release of the data to third parties. The lack of generalisability from having a single data point means that there is little value in releasing the data anyway.
    \item The study does not centre around sociological issues, and my participation in events will not be subject to judgement or inquiry. Simply, social issues are tangental to the content of the study (though it is notable that some readers may find more interest in them than the author intended).
\end{itemize}


The BNC, when building its speech corpus, followed a procedure whereby consent was sought but only after each conversation:
\begin{quote}
All conversations were recorded as unobtrusively as possible, so that the material gathered approximated closely to natural, spontaneous speech. In many cases the only person aware that the conversation was being taped was the person carrying the recorder.

For each conversational exchange the person carrying the recorder told all participants they had been recorded and explained why. Whenever possible this happened after the conversation had taken place. If any participant was unhappy about being recorded the recording was erased. During the project around 700 hours of recordings were gathered.
\end{quote}

Though this is a higher standard of consent than sought for this study, we feel this is offset by the fact that our data is private and  reviewed only by the subject himself.  This is a justification reinforced by the human rights act, which guarantees privacy to prevent release of unsolicited recordings, but makes no restriction on the use of such data by a participant in the original conversation.



% ============================================================================
\subsection{Future Work}

In the long term, it is hoped that a greater understanding of the above may contribute to:

\begin{itemize}
    \item Methods for augmenting and rebalancing corpora by identifying the position of a subject within a larger corpus' population;
    \item A greater understanding of variance in terms of the populations being studied
\end{itemize}

From a sample of just one person, it is possible to use auxiliary data from existing sources to operationalise and reason about inter-person variability.  This may be done by cross-referencing a subject's demographic variables with those from an existing corpus, placing them in context and allowing comparison of his linguistic data to other groups (or to those within a given similarity). This technique can also be used to impute data from partially-sampled sources, creating a personal corpus by re-weighting existing samples.

Unfortunately many existing corpora are unsuitable for this process due to the limited availability of metadata (something that is also an issue for those constructing ``informal'' subsets).

% ---
Methodologically, it is possible to generalise the data-gathering procedures mentioned in this case study either through reduction of the population covered (the technique used here in extremis), or reduction in the linguistic fields covered (the technique used more typically in special-purpose corpora).  A combination of these two approaches may well lead to the technique being used for many questions.

Careful application of elicitation strategies such as questionnaires or source-specific tools like web usage monitors may be able to produce sufficient auxiliary data to resample larger corpora, however, these methods would need justification from repeated studies such as the one described here.
