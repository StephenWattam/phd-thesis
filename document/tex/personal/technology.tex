Conventional corpus designs were chosen to avoid practical challenges that were existant at a time when computerisation was in its infancy.  The application of new technologies to the problems of sampling offers a way to mitigate a number of these issues, making alternate designs possible.

Two main themes are notable in easing access to text in a usable form:

\begin{itemize}
    \item Many more documents are now produced and consumed in digital form.  This means they are accessible for automatic copying and processing by sampling software, often without any intervention necessary by the user.  As techniques for cataloguing, monitoring, and annotating documents improve these data become richer, often in ways that would benefit an end user (and thus a corpus builder).
    \item The abundance and ubiquity of portable technology such as smartphones lowers the difficulty of many existing sampling methods such as audio recording or photography.  Connectivity of these devices allows for easier movement of data and offloading of processing, even when in physically remote areas.
\end{itemize}


The former of these is well represented by the Web-as-Corpus movement in corpus linguistics, and many corpora include sections which have been sampled by distributing portable technology (originally tape recorders).  Of particular interest, however, is the value that we may extract by exploiting both to form a coherent sequence of events, and using that narrative to inform corpus annotation.

One community that has been using such techniques extensively is that associated with life-logging.  Life-loggers record, and often catalogue, their own activities and use of many different kinds of resource in everyday life for reasons of posterity, entertainment, or memoisation.










\subsection{Life Logging}
% Just as the state of transcription technology has limited acquisition of spoken corpora, so have the limitations on digitisation and format conversion limited the selection of written text to those formats that are already indexed, or in formats standard enough to transcribe and represent easily.   As modern computing devices have miniturised and become ubiquitous, both of these issues have become less significant, rendering larger amounts of text recordable without significant manual intervention.

A number of technologies have been developed as a result of the life-logging community's interest in multimedia records.  Life-logging is an activity that emerged slowly out of the principles of webcam shows and reality TV, and involves recording (and usually broadcasting) continuous information about one's life as it occurs.

Though most popular efforts started as a means for providing entertainment, methods used soon diversified and gained the interest of the information retrieval and processing communities.  Many projects have been started with an aim to catalogue and operationalise the huge stream of data each person creates, largely with a focus on aiding that person in their daily life, or aiding large organisations (such as defence forces) in management of resources and people.





% --------------------
Life-logging as a distinct activity is often considered to have started with Jennifer Ringley, who started broadcasting her entire life using webcams in 1996 ('JenniCam'\footnote{\url{www.jennicam.org}, currently down.}).  Her website proved successful for many years, gained significant media coverage, and in many ways can be credited with popularising lifecasting (broadcasting of one's life rather than simply logging it) and helping to define the most recent wave of life-logging efforts.  However, she did not start the practice of either life-logging in general, or life-casting.


Life-logging using less technical methods has arguably been performed by millions in the form of diaries.  Though informal, the value these can offer as databases is well-known to historians as they offer a narrative structure that is difficult to build from other sources. The case study detailed within this chapter uses such methods for exactly that reason.


More formal, detailed forms of diary could properly be called the first life-logs.  Of particular note in this area is the `Dymaxion Chronofile'\cite{buckminsterFullerPapers}---Buckminster-Fuller's attempt to document his own life, which consists of a series of scrapbooks recording his actions (and documents) every 15 minutes between the years of 1920 and 1983.

Buckminster-Fuller's efforts captured the last 63 years of his life in extreme, multi-modal, detail: capturing all correspondence, bills, personal notes and material such as clippings from newspapers.  This detailed record of his life is essentially unmatched in the pre-digital era, and would not be attempted again until the digitisation of many common tasks eased the process of capturing documents.


For the first lifecaster, we must turn to Steve Mann.  Mann began working on wearable computing in the early 1980s\cite{mann1994wearable}, focusing on video recording and head-up-displays, however, it is clear from photographs of his equipment that it could hardly be considered unobtrusive enough for sampling purposes (indeed, it would probably prove more troublesome than Buckminster-Fuller's 15-minute interruptions).

The posterity-oriented and academic streams of life-logging persue a parallel course from the mid nineties onwards.  The aforementioned success of JenniCam led to a number of copycats, and, eventually, services such as ustream\cite{ustream2014homepage} and Justin.tv\cite{justintv2014homepage}; both of which make lifecasting available to anyone owning a smartphone.  The consumer side of life-logging has also been rising in popularity due to products such as Narrative\cite{narrative2014lifelog}\footnote{Originally known as `Memoto'.} (which emluates Microsoft's SenseCam\cite{hodges2006sensecam}) and Google's Glass project (which, though not explicitly designed for life-logging, provides hardware well suited to it).  There is also an increasing interest in smart watches, which promise to further extend the simplicity of accessing smartphone features in a continuous, unobtrusive manner.

Academically, much more research was undertaken on uses of the data gathered.  This meant a greater focus not just on multimedia methods and streaming, but also other sources of data such as documents (digital or paper), location data, etc.  Microsoft's Gordon Bell is particularly well known in this area for developing SenseCam\cite{hodges2006sensecam}, which focused on photography and location data, and MyLifeBits\cite{gemmell2002mylifebits,gemmell2006mylifebits}, which consolidated many different sources.  In these cases (and others' similar efforts\cite{huynh2002haystack,dumais2003stuff,dittrich2006imemex}), the focus was on producing a record of life that could be used by the original subject as a form of super-accurate memory, similar to Vannevar Bush's `Memex' vision\cite{bush1945we}.

In one study, mechanisms for de-duplication are examined with a view to improving the interface used when browsing large collections of logged data\cite{gemmell2006clean}.  Therein Wang \& Gemmell develop a filter based on clustering that identifies not just identical documents, but also those similar enough to carry identical information in slightly different forms (for example, with updated boilerplate features or layouts):

\begin{quote}
\ldots{}we count both the exact duplicates and near duplicates we find, we can effectively reduce 21\% of all documents and 43\% of web pages from the viewer to reduce clutter of the user interface.
\end{quote}

A similar project, funded by DARPA (but later dropped) was simply called LifeLog\cite{allen2008dredging}.  This took a similar approach to MyLifeBits, aggregating many sources of data into a single bundle, organised by time.  Unlike MyLifeBits, however, much of the focus was on using information retrieval methods to construct a coherent narrative for a person, which could later be interrogated at a higher level than the collected data itself.

Work on `audio-based personal archives' by Ellis \& Lee\cite{ellis2004audiolog} moves the focus of much of this logging from visual to audio recording, with the intent being to create a digital memory of events that is based on the most inconspicuous and unobtrusive recording methods.  They present a number of audio analysis techniques, including those for anonymising data without loss of other information, that may be used for speech data.

Much of the focus of these projects was on recall and operationalisation---tasks that are particularly difficult and largely ignored by the more entertainment-focused communities.  It is unfortunate, however, that their purpose was largely one of narrative creation: many exploration and recall tools rely on a human's ability to interpret the complex data recorded, and this is ill-suited to the more formal sampling approach needed here.

One notable exception to this is Deb Roy's project to record the language use of his own child (most famously exposited in his TED talk\cite{roy2011ted}).  This involved continuous video recording using a number of cameras in his house.  The footage from this was then analysed to identify regions where his child was speaking, and thus develop a corpus for use in language acquisition studies.  He used a number of methods to identify interesting video regions that, it was initially hoped, would apply to analysis of the speech portion of this case study's corpus.  In reality, however, the conditions under which he records audio proved significantly simpler than those in this case study.



Because of the continuous nature of life-logging, efforts have been made to use methods that are easy to maintain, self-contained, and covert.  Due to this, as well as the original intent of the life-logging process, much of the effort surrounding life-logging focuses on multi-media sources, and how they may be best combined to form a coherent idea of context.  

Typical sources of data considered include:

\begin{itemizeTitle}
    \item[Video recording] Many life-loggers wear systems that are able to continuously record video in the direction they look, and upload this using mobile networking systems.
    \item[Audio recording] Due to its lower obtrusiveness, many efforts surround the analysis of audio logs, and include systems to detect voices and identify events such as making appointments.
    \item[Document storage] With the increase in use of born-digital documents, some of the more holistic life-logging systems record documents as they are read, with a focus being to integrate this data into the larger picture.  Others scan in physical documents such as their post for later retrieval.
\end{itemizeTitle}

Many of the requirements of a life-logging platform (covert operation, comprehensive data management, context identification) overlap notably with the methods used in covert sociological research and, of particular note for our purposes, those constructing spoken language corpora.  The distinction drawn here is one of philosophy.  Where life-logging focuses on construction of a narrative, contextualising data, more conventional sampling methods have focused on data recording and normalisation in a formal setting, discarding data that is not immediately operationalisable.


Notably, one of the methods used to create the spoken portion of the BNC was covert recording, where 124 people were provided with tape recorders\cite{burnard1995users}:
\begin{quote}
Recruits who agreed to take part in the project were asked to record all of their conversations over a two to seven day period. The number of days varied depending on how many conversations each recruit was involved in and was prepared to record. Results indicated that most people recorded nearly all of their conversations, and that the limiting factor was usually the number of conversations a person had per day. The placement day was varied, and recruits were asked to record on the day after placement and on any other day or days of the week. In this way a broad spread of days of the week including weekdays and weekends was achieved. A conversation log allowed recruits to enter details of every conversation recorded, and included date, time and setting, and brief details of other participants.
\end{quote}

As illustrated by the BNC's demographic balancing of that portion of their corpus, this ability to directly record data from the field satisfies the disadvantages of text-index-oriented methods of document selection, allowing us access to all of the contextual data at the time of text consumption/production (this is particularly advantageous for spoken texts, where production and consumption often occur soon after one another).

The cost of this approach is (as felt by all sociological studies) a need to find a sufficiently large and heterogeneous sample of people who may wish to record data about their language use (and for long enough for it to be useful to researchers).  This is arguably more difficult in practice than text-index-based methods, and should only be considered at a large scale where the difference is likely to be crucial to a study, nonetheless, large samples (such as the British Household Panel Survey\cite{taylor1996british} and the British Crime Survey\cite{hough1983british}) exist in sociology as testament to the value such designs may yield and the illustration that no alternative method exists for many sociological issues.

The ability to specify the demographic variables of a sample directly makes techniques using logging particularly applicable to the construction of special-purpose corpora, especially where those corpora are best demarcated along social lines.  Indeed, at one extreme of this scale exists the concept of a personal corpus; something that may yield insights or models about a single person's current language usage.  Such a resource may one day be particularly valuable in defining how one uses text-based interactive systems (such as the web) or reads content (such as news articles).

Such designs exhibit a tradeoff: a decrease in socioeconomic breadth in exchange for an increase in linguistic breadth.  It is the intent of this chapter to illustrate the value in this approach, and to investigate methods by which it is possible to construct corpora without undue difficulty through the use of life-logging methods.
% This needs mega elucidation, methinks




In practice the distinction between life-logging methods and `conventional' data recording for research is one of rigor and intent.  Life-logging is primarily concerned with capturing \textsl{all} data, often in a best-effort manner, using whatever means is necessary to unobtrusively do so---favouring imperfect but complete records.  Conventional data capture techniques, though often overlapping methodologically, focus on specific research questions; seldom recording data that cannot easily be operationalised and valuing quality recording of few variables, over opportunistic recording of many.

%i.e. As the methods life-loggers and life-casters use to record and organise their captured data progress, they often converge with existing research methods.


