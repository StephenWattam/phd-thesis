The degree to which a corpus represents ``real'' language use is arguably the most important aspect of corpus design.  General purpose corpora are constructed typically based on a mix of expert opinion and per-medium data sources such as bestseller lists.

% The degree to which a general purpose corpus represents ``real'' language use is often debated at length, and is the key issue affecting design of corpora.  A wide variety of methods have been used to estimate language proportions and usage for a whole population, however, these are mediated by (and often centre around), expert opinion.

Reasons for this approach are both theoretical and practical---sampling individuals in the act of using language is very difficult, and the persistent nature of texts means that they remain available for sampling after such ``usage events''.  This is especially relevant where the corpus designers intend to include older works.

Nonetheless, there is a missing empirical link between the expert-guided designs of language proportions within a corpus and the ground truth of an individual's (or population's) language use.  This is lamented by many who wish to study language acquisition, or apply more detailed models of language comprehension.

This chapter describes a case study where a census of language use was constructed for a single individual, described by genre and source.  This sample design yields very little inferential power about whole population, but serves to illustrate the disparity between the proportions a general purpose corpus may have, and the language used by an example individual.  This more closely follows Hoey's~\cite[p.14]{hoey2005lexical} model of describing the body of language a single user is exposed to:

\begin{quote}
Not even the editor of the Guardian reads all the Guardian, I suppose, and certainly only God (and corpus linguists) could eavesdrop on all the many different conversations included in the British National Corpus.  On the other hand, the personal `corpus' that provides a language user with their lexical primings is by definition irretrievable, unstudiable and unique.
\end{quote}

The extent to which this one subject may be a useful guide for corpus building is not addressed directly, but this is an area that could be expanded using demographic auxiliary data in order to fulfil at least some requirements of Hoey's personal corpus.

Though beyond the scope of this thesis, this approach could be generalised for use with questionnaires and other less-invasive data gathering techniques, or specialised and restricted to a given source of data for use with automated tools (for example, analysing a user's web history).  We hope that the methods described herein offer some value for this further work.


\paragraph{}
In Section~\ref{sec:personal:sampledesign}, we describe the design of the sample being built in the study, focusing on the key differences between the approach taken here and contrasting it to Brown-influenced corpus methods.

Section~\ref{sec:personal:technology} examines how the availability and ubiquity of technology has enabled people to mitigate practical issues with data gathering, and surveys the efforts of life loggers, whose goals often align well with those of this study.

Aims and objectives are covered in Section~\ref{sec:personal:aims}, and the sampling policy is specifically stated.

Section~\ref{sec:personal:method} describes how this sampling policy was implemented, and the design decisions affecting different technologies used to record data.  Each data source is identified and categorised according to its persistence and ease of sampling.  Later, in~\ref{sec:personal:method:operationalisation}, this section focuses on operationalisation and post-processing the recorded data.

Results are shown quantitatively in Section~\ref{sec:personal:results}, and the dataset is described with some comparison to the BNC\@.  The data is broken down by linguistic events and by words, and summarised.

Section~\ref{sec:personal:discussion} analyses this data qualitatively, drawing comparisons against the BNC with relation to the subject's demographics and to the activities performed within the sampling period.  This section also addresses how the data inform the method, evaluating key aspects of the data gathering process such as validity and ethical concerns.

Finally, we conclude the chapter in section 4.8 with a short review of the methods, and a discussion of how these contribute to the overall themes of the thesis.

