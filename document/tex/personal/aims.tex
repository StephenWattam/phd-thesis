
The case study described here is an attempt to assess the extent to which techniques from life-logging may assist corpus builders in creating a demographically-oriented corpus.

It follows an iterative design in order to gradually refine the methods used, focusing on:

\begin{itemize}
    \item The variables that may practically be recorded about a text (and that must be before they are lost);
    \item Methods that may be used to sample text unobtrusively, especially how new technologies may be used to assist;
    \item Methods and tools for operationalising logs after sampling is complete, and how these may ease the process of data gathering itself.
    \item How to minimise the intrusiveness of capture methods both to the experimentor (meaning that can capture smaller interactions) and to those around him (meaning the data is more representative).
\end{itemize}

Ultimately, the differences in sample design contribute to a larger picture that could yield much future work.  The issues addressed here are:

\begin{itemize}
    \item What methods may be used for gathering data in a short-term language census?
    \item How may the collected data be operationalised?
    \item What proportions of language are used by the subject; do these support common claims from general purpose corpora?
    \item How may these methods be used in future to aid those building corpora?
\end{itemize}


The case study described here is a first effort in exploring a method that may be useful to many fields.  Aspects of the life-logging approach to determining corpus properties could be generalised (or relaxed) in order to further reduce the demand on the subject.  This thesis does not permit further study on these methods, however, it is written with a view to clarifying further work into questionnaire-based or electronic elicitation of language proportions.






\subsection{Sampling Policy}
\label{sec:personal:samplingpolicy}
The aim of a personal corpus is to emulate, as closely as possible, a census of observed language.

Use of language, in this context, is defined as any conveyance of information, spoken or written, in any quantity. There are no bounds to the context in which it is used, nor the language itself, as the purpose of the study is to evaluate these very things.

Each of these transactions is recorded as a single line in the data set, and will be annotated with the variables recorded.  One of the major issues encountered in preliminary tests was annotation for attention and proportions read.  These will be recorded along with textual properties to ease operationalisation.

Attempts were also made to record sufficient data to retrieve the full text of each transaction.  This worked better for some data sources, and much of the case study thus concerns itself with (sometimes estimated) word counts.  Word counts were chosen as a measure of size due to their use in other corpora, and their applicability to many different media.


A number of practical challenges were identified before sampling, and these were backed up by experience:
\begin{itemizeTitle}
    \item[Review and Production] Both should be recorded as fully as possible. Where a text is re-read, or developed and continually re-read, this should be noted as an ongoing process (and accordingly oversampled).
    \item[Short Utterances] Very short interactions, such as passing greetings, should not be under-recorded. Their inclusion is likely to be one major difference from conventional corpora.
    \item[Oft-reread Texts] such as labels, signs and the like. It’s debatable whether or not one actually reads or merely remembers/recognises these. \til{Perhaps psychological literature (Gestalt, etc.) can shed some light on this last point?} 

            
            \til{ See Wang, Zhe, and Gemmell, Jim, Clean Living: Eliminating Near-Duplicates in Lifetime Personal Storage, Microsoft Research Technical Report MSR-TR-2006-30, March 2006. }
\end{itemizeTitle}
