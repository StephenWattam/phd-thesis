\section*{Introduction}
\label{sec:evaluation:introduction}


We have seen in previous chapters (\td{which}) a number of scientific and practical issues that affect current corpus collection methods.

This chapter details the design and implementation of a method for performing number of common corpus building workflows.  These workflows are designed both as a continuation of the methods detailed in Chapter~\td{chapter 4} and as part of work with existing corpora, and represent areas of scientific importance in corpus studies.




The chapter begins in Section~\ref{sec:rebuilding:rationale} with an outline of the workflows involved, detailing the rationale behind each, and the value that formalising the process may yield.  Immediately following is an outline of the design of the method, specifying the processes involved and comparing them both to existing procedures and to the original workflows.

Section~\ref{sec:rebuilding:method} describes the mechanisms used to implement these designs, and covers the architecture of the software used to test the method.  It discusses the algorithms used, as well as the variables selected for the test system and the strengths and weaknesses of these as a mechanism for evaluating the method described earlier.  This section then specifies the technical details of the implementation, and provides details on the approximation and optimisation methods used.  

Finally, the conclusion relates the design and implementation of the method back to the aims detailed in \td{ref chapter 2}, drawing comparisons again to the capabilities of other existing systems.






\section{Objectives/Research Questions}
\label{sec:evaluation:rqs}


The overall method used in this evaluation is one of treating the system to be tested as a method of replicating the features of a corpus, passing statistical properties of the input corpus through to the output.

The capacity of any system using auxiliary data to do this is dependent on the nature of the population, and how it relates to the input.  The data used for evaluation must then be a subset of the population of documents accessible using the search heuristics chosen---if testing with another seed corpus, one must change the search strategy modules to fit, changing the overall results of the tool.  Essentially, this is a manifestation of 'garbage in, garbage out': moving search strategies into the implementation merely makes this more explicit.

One method of testing this is to see the output corpus as a model of the input, given certain assumptions that include the selection of metadata types and search strategies: providing those assumptions hold, much of the variation in the input corpus ought to be explained by variation in the output.  Measuring deviance between the two allows us to identify specific areas of poor fit, which can then be improved either by altering the pluggable modules to fit the ground truth.

The purpose of this evaluation is to test both the search modules used for the particular corpora given, and the overall method: if no set of reasonable assumptions can be found, this is an indication either that the population of documents online is fundamentally different to that of the input corpora tested, or the method presented here is incapable \textsl{by design} of identifying appropriate documents.

% --

Research questions surrounding this method run to:

\begin{itemizeTitle}

    \item[Components] How valid are the assumptions of each of the retrieval methods and heuristics selected?

    \item[Overall Application] For general-purpose input corpora, to what extent (and in what manner) does the output corpus resemble the input `seed' corpus?

    \item[Feature Correlation] Do the differences between two input corpora match those between two corpora built using them as seeds?

    \item[Residual Variance] What consistent features remain variable between the output and input corpora, i.e. what data cannot be sought online using the heuristics/search methods selected.

\end{itemizeTitle}

The accuracy of each heuristic component is constributory to the excess dispersion in the output corpus.  By design these modules are unambitious, relying on existing methods and tools already tested in the literature, however, their performance upon the data used here will be evaluated in order to better explain sources of error.  As this system remains a proof of concept, the selection of these modules is limited.






\til{put in discussion: this restriction already exists for tools like bootcat, but without the explicit control over search mechanisms provided using the method being evaluated here.  Since there is no theoretical way around the `garbage in garbage out' problem, providing easy operationalisation to users is one approach to reducing overall methodological errors}

---notably, selecting a wildly different input corpus without changing the sources of data will result in wildly different 








\section{Performance of Heuristics}
\label{sec:evaluation:heuristics}
The heuristics selected for this evaluation are formed around Lee's BNC World index~\cite{lee2001genres}.  This selection was chosen because of their alignment to operationalisable, human-level metadata and the existence of multiple corpora with this level of annotation.



There are two main approaches to populating the corpus description using these heuristics: either read the seed corpus' contents and classify each data point, or read a list of metadata from an existing index.  The latter approach is used in this evaluation, since it is applicable to corpora with partially-missing data (such as the personal corpus data resulting from data gathering in Chapter~\ref{5}).

The accuracy of the classifiers listed here is responsible for minimising excess dispersion relative to the input corpus.  The nature of their residual error is also going to apply bias to the resulting data set.

Since many of these heuristics surround operationalising a corpus, a large body of research exists for classifying and extracting useful dimensions from texts.  The heuristics presented here are proof-of-concept only, and it is expected that the design of the heuristics used for a study is selected to match the theoretical basis of any analysis.

The heuristics presented here are document-level.  In most corpus designs, word count would be considered a measure of the size of the corpus (rather than a property of its constituents).  The method evaluated here is capable of retrieving truly IID samples at different levels, and demands a different selection of heuristics and metadata when operating at the word or sentence level.  Document level metadata are both high-level enough to be distributed for confidential corpora and descriptive enough to enable accurate retrieval (by contrast, word or part-of-speech frequencies would reveal much of the contents of the original corpus, which may not be desirable).

\subsection{Audience Level}
* Reading level used as an estimator
* Means, standard deviations computed from BNC
* Accuracy test with nearest-mean classifier

\subsection{Word Count}
* Compare word counts to BNC files
* Talk about boilerplate removal

\subsection{Genre}
* Classifier description
* Talk about rationality of errors
* Talk about how difficult this is
* Evaluate unigram, ngram, naive bayes
* Comment on the choice of genres, how artificial or simple ones might be better (perhaps aligned with data source such as DMOZ?)





\section{Method}
\label{sec:evaluation:method}
Discuss the evaluation methods.  Firstly talk about the error monitoring methods built into the program, then detail the linguistic feature analysis.  This is the main body of the analysis of the method and doesn't cover the heuristics


\section{Data}
\label{sec:evaluation:method}
Discuss which corpora were chosen and why

\section{Results}
\label{sec:evaluation:}
Provide quantitative measures for the two evaluation methods
\subsection{Partial/white box}
Show how well classifiers and heuristics behave
\subsection{Complete/black box}
Show how well the whole system works


\section{Discussion}
\label{sec:evaluation:discussion}
Talk about the various strengths/weaknesses w.r.t. the original RQs and the goals in ch5.
\subsection{Choice of submodules}
talk about which heuristics seem good, and what needs improving


\section{Summary}
How well does the system work w.r.t. chapter 5 aims and objectives, and how well does this chapter 
answer its own RQs.


\section{Further Work}
\label{sec:evaluation:furtherwork}
Cover the heuristics and modules that need making, and possible tweaks to the error feedback mechanism.  Also comment on the knowledge we have of various populations online and how this impacts inherited bias.

Link to an appendix with the generalised form of the proxy corpus thingy?
