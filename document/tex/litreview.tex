Large samples of text, known as corpora, are fundamental to a great number of users.  The groups I will be focusing on for the purposes of this thesis are particularly interested in scientific samples, that is, in generalising findings from the corpus to a wider population.  This includes much work in NLP and corpus linguistics, and some of what is commonly termed `data science'\footnote{Though in most `data science' contexts, sampling is performed as an intrinsic part of the task}.


The problem of sampling linguistic data is a well-researched one in the field of corpus linguistics, and a review of linguistics' approach to corpus construction will form the bulk of this review.  This portion of the review begins with a historical overview of approaches to corpus construction, before focusing on the problem of defining a corpus in concrete enough terms to suit the rest of the work presented in the thesis.  It then focuses on potential threats to validity exposed by these methods, both from the point of view of corpus construction and use.

Not to be neglected, however, is the perspective offered by more formal survey sampling, especially as its use across the social sciences can offer insights into solutions for practical problems.  Each area of sampling is approached as if applied to corpus construction, and its relative concepts identified.

Finally, in~\ref{sec:litreview:newtech} the first few steps towards improving sampling methodology within corpus linguistics are reviewed, with a focus on web-based corpus construction methods.




\section{Corpora and Corpus Linguistics}
\label{sec:litreview:corpora}


The field of linguistics is one concerned with description and formalisation of a particularly ethereal social concept.  The paucity of philosophical agreement upon the nature of language has led to many different approaches being taken through the years, many of which have accomplished great things in advancing our capacity to reason about, and derive conclusions from, language.


The most obvious method for inquiring about the nature of language is to sample real-world use.  This process is followed in many other sciences concerned with social phenomena, and offers a tried-and-tested methodology for inferring results.  It is perhaps unsurprising, then, that this method has been used to a varying degree by many linguists throughout history.

In their 2001 book\cite{macenery2001corpus}, McEnery and Wilson characterise the pre-Chomskian linguistic inquiry as primarily taking this form:

% page 2
\begin{quote}
The dominant methodological approach to linguistics immediately prior to Chomsky was based upon observed language use.
\end{quote}

They point to a number of studies using systematic analysis of samples to make their conclusions\cite{kaeding1897haufigkeitsworterbuch,preyer1889mind,stern1924psychology,eaton1940semantic,west1953general}, and decribe Chomsky's influences on the field, which served to lead it to all but abandon corpus techniques in the 1950's, as a turn towards explanatory, rationalist theories of language.

These rationalist theories were often verified using experimentation or elicitation, in an effort to gather data that is detailed and reliable.  It's arguable just how valid and philosophically defensible these methods are, especially given the nature of language as a part-mental, part-tangible concept.

% These early corpora are of a quite different form to those of the `modern era', and will not be discussed in any great detail here except for historical context.


In a pre-digital world, collection and analysis of large-enough-to-be-useful samples of language was extremely difficult, making this rationalist approach a viable alternative.
Small samples are fundamentally unable to reveal some of the details examined by the structuralists, and construction and analysis of sufficiently large corpora in a non-digital era would prove a practical hard limit on the power of corpus studies.  To frame the focus on rationalist inquiry as an alternative to the empirical is a disservice to both: empiricism had supported the rationalist theories of the 50's, and would itself go on to be supported in turn as it once again rose to prominence.

The revival of corpus linguistics may be attributed almost solely to the availability of programmable computing machinery.  This offered a solution to the problems of scale encountered during earlier corpus analyses, making empirical data once more viable for detailed inspection of language.

This revival, often termed the `modern era' of corpus linguistics, has yielded what we would commonly call a corpus today: a large, machine-readable, annotated collection of texts sampled in order to represent some population of real-world language use.

Corpus studies are now widely relied upon across linguistics, and are often the method chosen to test theories derived from structuralist approaches.  This may be seen as a validation of their methods, as the two complementary philosophies of scientific inference are once more able to use one another without methodological suspicion.

For the purposes of this section I will be focusing on the design of `modern' corpora with respect to their use in validating linguistic theories, and for training automated Natural Language Processing (NLP) and Information Retrieval (IR) systems.  For this reason I will be covering mainly general purpose corpora: those that purport to represent such a large population as to cover a whole language.  This type of corpus is built so as to be useful to many research questions and researchers, in part to dilute practical issues surrounding sampling.  This is in contrast to special-purpose corpus, which are designed to represent a restricted context.  Special-purpose corpora may be selected according to demographic or linguistic properties, and are typically much smaller.  Because of this they are often built for a given study, or by re-sampling a general-purpose corpus.




\subsection{A Brief History of Modern Corpora}
The Brown Corpus of Standard American English\cite{francis1961brown}% http://icame.uib.no/brown/bcm.html
is widely regarded as the first of the modern age of corpora.  Built in the 60's, Brown's corpus was the first electronic-format general purpose corpus and was roughly one megaword in size.  % 1,014,312
It contains 500 samples, each comprising roughly $2,000$ words, that were taken to represent a cross-section of works published in the United States in 1961.  The proportions and sizes of samples were selected in order to trade off pragmatic concerns with the possible kinds of analysis that could be performed at the time.

The `Standard' in its name referred to Kucera and Francis' intent that the corpus represents their judgement of `standard' English use.  Brown became a \textsl{de facto} standard for American English, and the design was carried forward into many other corpora, mostly regional versions designed to be comparable to Brown\cite{hundt1999manual,johansson1986tagged,hundt1998manual,shastri1988kolhapur,collins1988australian,bauer1993manual,mcenery2004lancaster}.  In order to maximise the value of comparisons within studies, other general purpose corpora chose to mirror Brown's sampling policies.

% --- 

\begin{table}[Ht]
    \centering

    \begin{tabular}{llrr}
    \hline
    Categories & Texts in each category & American corpus & British corpus \\ \hline
    A & Press: reportage & 44 & 44 \\
    B & Press: editorial & 27 & 27 \\
    C & Press: reviews & 17 & 17 \\
    D & Religion & 17 & 17 \\
    E & Skills, trades, and hobbies & 36 & 38 \\
    F & Popular lore & 48 & 44 \\
    G & Belles lettres, biography, essays & 75 & 77 \\
    H & Miscellaneous & 30 & 30 \\
    J & Learned and scientific writings & 80 & 80 \\
    K & General fiction & 29 & 29 \\
    L & Mystery and detective fiction & 24 & 24 \\
    M & Science fiction & 6 & 6 \\
    N & Adventure and western fiction & 29 & 29 \\
    P & Romance and love story & 29 & 29 \\
    R & Humour & 9 & 9 \\ \hline
      & Total & 500 & 500 \\ \hline
    \end{tabular}

    \caption{The basic composition of the British and American corpora}
    \label{table:litreview:corpora:lobdist}
\end{table}


The Lancaster-Oslo-Bergen (LOB) corpus\cite{johansson1986tagged} was built as a British counterpart to Brown.  
It uses the same stratification and sampling strategy (with one or two more texts in certain categories) and thus comprises roughly a megaword of British English, as published in 1961.  % See  http://clu.uni.no/icame/manuals/ Table 1 for a table of comparisons
Though the manual does note:

\begin{quote}
    The matching between the two corpora is in terms of the general categories only. There is obviously no one-to-one correspondence between samples, although the general arrangement of subcategories has been followed wherever possible.
\end{quote}

Table~\ref{table:litreview:corpora:lobdist} shows the proportions of texts in each genre, relative to Brown (the `American corpus'), as reproduced from the corpus manual.

% ---
The London-Lund Corpus\cite{greenbaum1990london} was released in 1990, and contains transcribed spoken text, annotated with a number of different markers to indicate intonation, timing and other extra-textual information.  
The corpus consists of $100$ texts, each of $5,000$ words, totalling $500,000$ running words of spoken British English.
The annotation scheme used in LLC is much more in-depth than that in many written corpora, reflecting the different nature of research questions for speech.



% --- 
Collins' requirement for a corpus upon which to base their dictionaries spawned the COBUILD and its `representative' subset, the Bank of English\cite{Jarvinen1994AMW991886.991985,sinclair1987looking}.  COBUILD uses a slightly different approach to corpus building: that of the monitor corpus.  Monitor corpora are continually added to, using a fixed sampling policy but an ongoing sampling process.  At the time of writing, the BoE is 650 million words in size (The whole COBUILD family used by Collins is 4.5 billion)\cite{collinswebcorpus}.

The approach taken by the BoE opens many possibilities for analysis of language over time (something also covered by diachronic/historical corpora using more conventional sampling).  Even so, such comparisons are complicated by the irregular additions (c.f.\ diachronic corpora, which contain complete samples for each time), and this remains the only major corpus built in this fashion prior to automated web retrieval.


% ---





The de-facto standard of the day is currently the British National Corpus, which comprises $100$ million words of British English\cite{leech1993100}.  The BNC's design was influenced heavily by discussions on corpus building that centred around creating a standard, reliable approach to taxonomy, sampling and annotation that occurred around the early nineties.

The BNC aims at being a synchronic `snapshot' of British English in the early 1990s.  It consists of samples of text up to $45,000$ words each, and is deliberately general-purpose, containing a wide range of genres as well as a sizable spoken portion.  It was released in 1994, but has since been re-coded and augmented, particularly notably by Lee\cite{lee2003bnc}, who constructed a significantly more detailed (and principled) taxonomy for its texts in 2003.

Lee's genre taxonomy is outlined in Appendix~\ref{sec:appx:sample}.  Lee's decisions to code genres using this system were partially based on fostering compatibility with the ICE-GB\cite{greenbaum1996international} and LOB\cite{johansson1986tagged} corpora.

The prominence and `whole population' coverage of the BNC's sampling frame spawned many compatible corpora, though these often show more variation in sample design than the Brown clones.  Xiao, in his survey of influential corpora, notes the existence of national reference corpora for American, Polish, Czech, Hungarian, Russian, Hellenic, German, Slovak, Chinese, Croation, Irish, Norwegian, Kurdish, and many more\cite{xiaoz2008}.  An updated version, BNC2014\cite{cambridgeuniversitypress2014}, is currently being constructed.

% 1.2 General definitions ( from http://www.natcorp.ox.ac.uk/docs/URG/BNCdes.html )
% 
% The British National Corpus is:
% a sample corpus: composed of text samples generally no longer than 45,000 words.
% a synchronic corpus: the corpus includes imaginative texts from 1960, informative texts from 1975.
% a general corpus: not specifically restricted to any particular subject field, register or genre.
% a monolingual British English corpus: it comprises text samples which are substantially the product of speakers of British English.
% a mixed corpus: it contains examples of both spoken and written language.


% ---
The rise of electronic communications has led to a reduction in the effort required to gather corpus data.  This has resulted in a great increase in the number of special-purpose corpora built for specific studies\cite{westlabusenet2013,baroni2006building,Mair20060101T00000009215034355}.  These corpora are more focused in that their construction methods restrict them to electronic data, yet  their large sample size may make them suitable for the study of smaller-scale features in a more general context.


% Move? [These corpora are often less widely used, and serve to illustrate one extreme discussed below, that the purpose of a corpus is important to its construction]

This thesis is focused on sampling using automated, technical mechanisms to overcome some of the challenges facing conventional methods, and as such focuses on web-based methods (known as `Web as Corpus').  WaC is concerned with sampling the web itself, as well as constructing samples that are representative of other data, yet are retrieved primarily from the web.  This approach, and those using similar methods, are covered below in Section~\ref{sec:litreview:newtech}.






















% ---------------------------------------------------
\subsection{What Makes a Corpus?}
The use of general purpose corpora as large monoliths, reused in many studies and systems, has led to much debate over the nature of a corpus.  This, as we shall see throughout the thesis, is a question unworthy of concrete answers---each purpose will exert certain demands upon corpus design criteria, and any widely-used corpus is likely to be a compromise around these.

This section exists primarily to define the term `corpus' as used here: it is unlikely that the definition derived below is universal, however, it is designed to reflect most use-cases, across corpus linguistics and NLP.

% This discussion is separated to cover a number of properties of corpora, and their parameters.  These 



In order to establish the important traits of corpora, it is wise to have an understanding of the motivation behind their existence.  Corpus methods are generally contrasted against two other methods of linguistic investigation: direct elicitation from a language speaker, and directed research into a linguistic feature under controlled conditions.  Both of these pose significant scientific challenges---both are reliant on at least one linguist's intuitive view of language (one that could hardly be said to be representative of most language users), and both require the acquisition of data without its usual context, something that is especially difficult given the varied and context-dependent complexity of language.

Corpora provide limited solutions to both of these issues.  In the former case, they provide an objective record of linguistic data that is free from all but the initial builders' influences (which, in the ideal case, may be documented and provided along with the data itself).  In the latter, they are as portable as any large volume of text, and may be annotated with context sufficient for a given linguistic task.


%------
At its most basic level a corpus is a sample of text.  Given research questions surrounding a body of text, it is perhaps necessary only to stipulate that a corpus must be represent known population~\cite[p. 22]{mcenery2001corpus}:

\begin{quote}
\ldots{}a body of text which is carefully sampled to be maximally representative of a language or language variety.
\end{quote}

Further to this, the modern definition of a corpus has undergone a series of significant refinements thanks largely to the increase in both the ubiquity and power of computing machinery.  General-purpose corpora are, with very few exceptions, electronic (with an increasing number documenting texts of electronic origin), multi-modal (covering a wide variety of methods of communication and their linguistic features), and annotated with linguistic data.

Many authors go further by stating that a corpus should be machine-readable, annotated with information useful to linguistic inquiry, built for a specific purpose or methodology, available for use in other studies, finite in size or stratified to provide multiple possible analysis methods with internally valid data\cite{bennett2010using,bowker2002working,leech1992corpora}.
%The majority of present-day corpora are “balanced” or “systematic”. This means that the texts are collected (“compiled”) according to specific principles, such as different genres, registers or styles of English (e.g. written or spoken English, newspaper editorials or technical writing); these sampling principles do not follow language-internal but language-external criteria. For example, the texts for a corpus are not selected because of their high number of relative clauses but because
Whilst I do not consider many of these to be requirements for a scientifically useful corpus, they may contribute greatly to corpus utility due to their alignment with common methodologies and uses.  There is undoubtedly a case for this---the utility of a corpus is often limited by its format, however, the extent to which this applies varies wildly by purpose makes it unreasonable to include many resources missing some of the above under the term `corpus'.


%Awesome quote from http://www.linguistics.ucsb.edu/faculty/stgries/research/2009_STG_CorpLing_LangLingCompass.pdf
%However,
%since corpus data only provide distributional information in the sense mentioned earlier,
%this also means that corpus data must be evaluated with tools that have been designed to
%deal with distributional information and the discipline that provides such tools is statistics.
%And this is actually completely natural: psychologists and psycholinguists undergo comprehensive
%training in experimental methods and the statistical tools relevant to these methods
%so it’s only fair that corpus linguists do the same in their domain. After all, it would be kind
%of a double standard to on the one hand bash many theoretical linguists for their presumably
%faulty introspective judgment data, but on the other hand then only introspectively eyeball
%distributions and frequencies. 

This review, then, shall start with the most basic of definitions, that of `a body of text sharing some property that may be interrogated for some linguistic information'.  This takes into account the claims of generalisability that cannot be made for more haphazard collections of texts (often called \textsl{libraries} or \textsl{archives} with no relevant common features) which are not demarcated by the boundaries of some notional property.










\subsubsection{Representativeness} % and Transferrability
Representativeness is, in effect, the goal of any sample.  It is the property that I chose to use as the loose starting condition above, and it is to be maximised by selection of those properties covered below.  By virtue of this holism, it is also dependent upon enough factors to be poorly defined in the literature.
% (by virtue of the difficulty of doing so).

The concept of representativeness is based strongly on philosophies of inference, and epistemology in general.  These are highly dependent on correlation in variables external to those measured by a sample, and any rigorous definition will involve the properties we wish to generalise, the purpose of such generalisation, and the population we wish to generalise about.  Users of reference corpora (those general-purpose corpora designed to represent a whole language) may find, for example, that their claims to validity are wildly variable compared to previous studies, simply due to the nature of their research question.


Much has been written on the subject of representativeness\cite{biber1993using,varadi2001linguistic,varadi2000corpus,leech1992corpora,rapp2014using}, both in linguistics and in other fields that are dependent upon complex sources of data.  As with psychology or sociology, the opacity of the mechanisms that generate language is such that significant philosophical disagreement as to the underlying nature of the data occurs.  This disagreement has, in many ways, limited efforts to formalise and reason about representativeness in corpus design: taken quantitatively, one person's adequate corpus is another's woefully biased one.

The LOB manual hints at the fluid nature of ``representativeness'' in corpus linguistics, and the degree to which corpus design is expert-guided\cite{johansson1986tagged}\td{can only find online version, no page nums}:
\begin{quote}
The true “representativeness” of the present corpus arises from the deliberate attempt to include relevant categories and subcategories of texts rather than from blind statistical choice. Random sampling simply ensures that, within the stated guidelines, the selection of individual texts is free of the conscious or unconscious influence of personal taste or preference.
\end{quote}

I consider this a false dichotomy: a high quality random sample is essentially the gold standard to which expert designs should be aspiring, and both are aiming for the same notional goal.  The key benefit to random sampling is, as observed above, the immunity from `unconscious' bias\cite{barnett1991sample}. %p.19

This ambiguity could be seen as an argument against the concept of a general purpose corpus, however, current progress indicates that this would be hasty: whilst special purpose corpora are often burdened by fewer procedural and pragmatic difficulties, they are necessarily limited in scope.  It is still rational and useful to identify speakers of a language as a homogeneous group at some level, especially for smaller linguistic features.  On the other hand, only a sample that properly encompasses a large amount of variation may be used to describe many effects of interest.

With this in mind, many of the studies into representativeness have focused on examining existing corpora, with a view to testing their internal variation against some known re-sample.  This approach has been taken most famously by Biber\cite{biber1993representativeness}, who performed a series of studies in which he compared the content of 100-word samples from the LLC and LOB corpora.  Each sample was paired with another from the same text (LOB samples points are $2,000$ words each, and LLC's are $5,000$).  Biber went on to extract some small-scale linguistic features from each sample, before examining the difference in frequency between each.

Biber concluded that existing corpora were sufficient for examination of smaller linguistic features, however, in the process he saw fit to reject the notion of representativeness used here (and, notably, everywhere else)~\cite[p. 247]{biber1993representativeness}:

\begin{quote}
Language corpora require a different notion of representativeness, making proportional sampling inappropriate in this case. A proportional language corpus would have to be demographically organized\ldots
%(as discussed at the beginning of Section 3.2),
because we have no a priori way to determine the relative proportions of different registers in a language.
\end{quote}

Biber's argument is that we should not aim for any degree of proportional representation of linguistic features, for this would produce a corpus that is mostly one kind of text (due to a conjectured Zipfian distribution of text types).  In many ways, this is an argument reflected by stratified sampling with artificially boosted proportions, however, stratification of samples (rather that with adjusted weights) is normally performed as a compromise, where we are aware from previous sampling efforts that we cannot adequately sample randomly.  Biber's presentation of this idea has seemingly led many to reject the common wisdom of sociological sampling, leading to corpus building to be seen as a fundamentally new activity, and consequently as something of a black art.
% black art within the community.


%-----
A further aspect clouding the waters of quantitative representativeness assessment is disagreement over how to parsimoniously stratify language.  This is in part due to the difficulty in defining a taxonomy for genre\cite{lee2001genres}, which is often the most controlled variable within a general purpose corpus' sample design.  Ultimately, I consider that the answer to this is, as mentioned in the LOB manual, down to the individual research question: a corpus sufficient to represent one feature may easily be massively biased for those with greater variation in the population.

%-----


In part for the reason that Biber's work was taken early on as a validator of current practices in corpus building, the notion of representativeness has remained an almost entirely philosophical concept within corpus linguistics.


%----


The BNC, though often relied upon as a reference corpus, makes a number of bold assumptions regarding representativeness---to the point of explicitly stating its deviation from being a solid representation of its population~\cite[p.6]{lou1995users}:

\begin{quote}
There is a broad consensus among the participants in the project and among corpus linguists that a general-purpose corpus of the English language would ideally contain a high proportion of spoken language in relation to written texts. However, it is significantly more expensive to record and transcribe natural speech than to acquire written text in computer-readable form. Consequently the spoken component of the BNC constitutes approximately 10 per cent (10 million words) of the total and the written component 90 per cent (90 million words). These were agreed to be realistic targets, given the constraints of time and budget, yet large enough to yield valuable empirical statistical data about spoken English.
\end{quote}

This implies that the corpus should not be taken as a single sample at all, or that large adjustments should be made during analysis to avoid generalisations on purely quantitative bounds.

A similarly pragmatic approach is taken to the sampling of written material according to production and reception statistics, with a balance being brought between the two based on publication, library lending, and magazine circulation statistics.  Further, some samples were taken purposively~\cite[p.10]{lou1995users}:

\begin{quote}
Half of the books in the ‘Books and Periodicals’ class were selected at random from Whitaker's Books in Print 1992. This was to provide a control group to validate the categories used in the other method of selection:  the random selection disregarded Domain and Time, but texts selected by this method were classified according to these other features after selection.
\end{quote}

The spoken portion contains a `demographically sampled' portion (comprising 50\%), so called because it makes an attempt to represent speakers according to their sex, age, and social class.  Individuals were randomly distributed, and each provided recordings of their conversations over a two to seven day period.  The major limitation noted here is the short period of time, and thus the low probability of capturing certain, rare, interactions on tape.  In order to remedy this, half of the spoken component was devoted to genre-based stratification, with the intent being to capture a greater breadth of data.


\til{Include NLP representativeness stuff?
\\
So, I definitely found some papers back in 2013, but can't dig them out now.  Is this todo droppable?
}








\subsubsection{Size}

The size of any sample is a crucial and often hard-to-determine property, and corpora do not differ in this respect.  In many ways the question of corpus size is a primary constituent of the representativeness property above, and though it has long been recognised as such there remains little consensus on just how large is large enough.

This disagreement is in part because language exhibits a number of properties that make it hard for us to gauge variability in the population, meaning that most sample size estimation methods (which rely on random sampling) are poorly suited.

%-

The first of these is the inability to accurately measure the complexity of language, which, as used and applied by humans, has unknown degrees of freedom.  This effect applies itself at many levels:

\begin{itemize}
    \item Lexicon---Vocabulary, even when restricted to a given demographic, time period or person, is difficult to define with any certainty.  Many people are capable of recognising entirely novel words, simply by virtue of their context or morphology (and the inverse, e.g.\ \textsl{Jabberwocky}).
    \item Syntax---The meaning of words and phrases is heavily context based, but the context affecting each aspect of language is variable and, in some cases, wide-reaching.  This makes it hard to determine if we should be sampling $2,000$-word texts, single sentences, or the whole thing\cite{hoey2005lexical}.  This unknown sampling unit drives one of the key trade-offs in corpus design, as a 1-million word corpus comprised of single sentences will be capable of embodying more variation than will one made of two $500,000$-word samples.
    \item Semantics---Variation in our understanding of language in context is poorly understood.  This is the driver behind any models of the above, but also affects how we should sample external variables such as socioeconomic factors and even the direct circumstances in which a text is used\cite{sinclair1991corpus,Stubbs19950101T0000000929998X23}. % sinclair p70 for collocation stuff
    %\item Discourse---
    %\item Pragmatics---
\end{itemize}

This ambiguity produces a tradeoff that has been identified by relatively few in the corpus-building community\cite{evert2006random,kilgarriff2005language,gries2011corpus}\td{Which ref from Baayen?  The one I have in bibtex doesn't seem suitable}% \td{(notably biber, Greis, Evert with the library)}
---that corpora of equivalent size may yield significantly different inferential power due to their differing number of data points.  This may be seen as an issue of depth vs.\ coverage: corpora including large snippets of text are capable of supporting more complex models and deeper inference, at a cost to the generality of their results.

There remains disagreement upon the extent to which these two aspects should be traded off, though it is notable that the NLP and linguistics fields vary greatly in their treatment of the data, with linguistics typically focusing on frequency and immediate collocation, and NLP being skewed towards more complex, instrumentalist, models.  %\td{cite? how?}
I would suggest that, realistically speaking, researchers should be examining their experimental design with respect to the size and/or complexity of the features they are working with in order to select a corpus that matches most closely.  It is also quite clear that this does not happen: the BNC contains a small number of large samples, yet is often used to analyse small-scale linguistic features.

%- 

Secondly, establishing the boundaries of a given language is difficult.  Users of a general-purpose corpus wish for two conflicting properties to be satisfied---the population must be wide enough to provide a useful set of persons about which to infer (and, more loosely, this should align with those persons we can say informally, for example, `speak English'), yet the corpus must provide sufficient coverage to represent that population in the first place.

Generally it seems that this problem, though having been recognised, has received too little effort for practical reasons.  Issues of balancing demographics in corpus design have typically taken a curiously detached form, that of selecting a sample of language as a proxy for demographics (e.g.\ selecting the bestsellers over unpopular books or sampling more popular newspapers).


This approach has led to a situation where each and every general-purpose corpus carries with it the expert judgement of linguists not only in selecting a wide variety of texts from within the population, but also their socio-linguistic opinions.  Sinclair makes this explicit in `Developing Linguistic Corpora', where representativeness is described in entirely subjective terms~\cite[p.5]{wynne2005developing}:

\begin{quote}
A corpus that sets out to represent a language or a variety of a language cannot predict what queries will be made of it, so users must be able to refer to its make-up in order to interpret results accurately. In everything to do with criteria, this point about documentation is crucial. So many of our decisions are subjective that it is essential that a user can inspect not only the contents of a corpus but the reasons that the contents are as they are.
\end{quote}


This reliance on expert opinion to overcome practical challenges associated with text retrieval has lead to the sampling policy being somewhat opaque to end users.
Those using a given corpus are not necessarily able to rigorously define about whom they infer a given result.  In practice, this manifests as a need to qualify results by reasoning about the likely impact of any ambiguity.


%-

Finally, disagreement on how we should extract features from language samples (i.e.\ which dimensions of variability are interesting) means that, aside from the immediate and obvious properties such as genre (about which there is arguably less agreement\cite{lee2001genres,aston2001text,sharoffs2015}), any efforts to stratify language are met with suspicion.  This may lead to researchers performing corpus analysis using informally-subsampled general-purpose corpora, with questionable correspondence between the categories used to select texts and their research goals.
It is the author's opinion that this is unanswerable except for individual studies: in order to know variation in features affecting one's use of a corpus, it is necessary to define the covariates and evaluation strategy.


%-

Further to these problems of defining the nature of the sample, there is the resultant problem of determining a sample size even where these are known.  

Taking the first of these issues in the extreme, it may be said that the only ``fully representative'' corpus must contain all context for each text (something that would include at best an abridged history of the world) in order to satisfy inquiry from many different perspectives.  Given the limitations in analysing properties of language beyond a given scale, and the clear impracticality of extending that scale's upper bound, it seems reasonable to conclude that current corpus efforts are sized so as to be useful for small-scale features, and that our inspection of language is currently somewhat shallow.

Taking into considerations problems of proportional, stratified sampling, it seems possible to establish a corpus size and composition that is widely agreed upon.  Nonetheless, issues of selecting a sampling unit mean that the resultant corpus may either be a refinement of current efforts (not necessarily a bad thing) or utterly colossal and beyond the capacity of even modern corpus processing systems.

Since sample size and composition are intimately related both to one another and to the concepts of representativeness and transferability, this issue will be one of primary importance to the rest of the thesis.





\subsubsection{Purpose}
The reason for sampling a given population is a crucial feature of corpora.  Not only does it define the level and type of metadata available (and the arbitrary definitions used therein), the expert selection of sample designs means it has a large influence on the sampling frame used, and thus the validity of any generalisations made using said data.

The condition given by McEnery \& Wilson\cite{mcenery2001corpus} here is that a corpus must have been built with some degree of linguistic inquiry in mind, for example, the collected works of Shakespeare would be counted, but not one's personal book collection (even if it happens to include the complete works of Shakespeare and nothing else).  This distinction seems to be little more than a way of stating that one's ability to infer things from a corpus is relative to its external properties, which is true of any sample.

This requirement calls into question two things: the generality of a corpus, and the extent to which it is documented.

General-purpose corpora sample a large population, of interest to many users.  This requires that they remain fairly unbiased and cover a large amount of variation (which in turn necessitates very large sample sizes).  Since they are re-used many times, the quality of their documentation is key to their gross value---each user will require particular variables in order to generalise from the text.

Special purpose corpora avoid this challenge by being re-used for less disparate aims (and, generally less frequently).  Because of the relative focus, their documentation is often able to be significantly more detailed, allowing for deeper insight.  This approach, however, is not transferable to the sample sizes used in general purpose corpora.

To contextualise this, both examples above merely constitute special-purpose corpora, that offer answers to different research questions.  For one, we may answer those about the nature of Shakespeare's language use, whereas the other allows us to find knowledge about a given person's literary preferences.  It is of no direct consequence that more people are likely to care about the former.

If we extend the example to include others' book collections, the ability to generalise changes: we know significantly more about Shakespeare than many other authors, and it is thus possible to annotate the Shakespeare corpus with details of his life, times when works were written etc.  The same could not be said of a corpus where our aim is to investigate reading habits (even if some people exclusively read Shakespeare).

% --- 
This `purpose' requirement can be phrased entirely in terms of external variables.
A group of texts about which nobody knows anything do not offer any opportunity for inquiry (except about themselves), and so it is reasonable to require documentation of the context in which those texts occur (or any external property that demarcates a homogeneous group).

By stating that a corpus must have a defined purpose, we include a set of reasonable assumptions about the corpus and its external properties: a corpus built for study of Italian newspaper text is unlikely to heavily sample The Guardian, for example.  In many ways, statement of purpose offers a shorthand for many decisions and assumptions inherent in construction of a large corpus, and a simple way to assess how well matched a corpus may be for another---related---purpose.

It is noteworthy, however, that selecting a corpus by its original purpose greatly complicates any subsampling that is possible---one must be able to subsample texts not only according to external data of interest, but also taking into account the original interactions and assumptions made by the constructors.  As mentioned above, it is unlikely that any corpus is capable of being documented `fully' enough to avoid these issues, however, this is a compelling argument for focus on detailed metadata being provided (rather than detailed rationales for sampling), especially in the case of general-purpose corpora---say `what', rather than `why'.


% see http://www.natcorp.ox.ac.uk/docs/URG/BNCdes.html


%----


\subsubsection{Data Format}
Some authors stipulate that a modern corpus should be electronic.  To generalise this position, they require that it is in some way processable by machines, i.e.\ that it must be in some way regular.
This defines the format of not only the basic textual content, but also the availability of metadata at all levels (category, document, word).

Both of these have been addressed to some extent, especially the problem of representing the text itself, which has largely been solved by UTF-8 at the character level, and XML/SGML at the markup level.  Standards derived from efforts such as the TEI\cite{ide1995tei} have some penetration, though there is often still the need to include proprietary extensions if other concerns on this list are to be maintained.

Though earlier work on corpus construction focused on data formats\cite{atkins1992corpus,EagTcwgCtypeaglespreliminary}, with a view to sharing corpora for local analysis, modern approaches tend towards corpora `as a service'\cite{hardie2012cqpweb,ferraresi2008introducing}---providing a front-end to query and analyse text directly whilst still hosting it at the original institution.  This approach is largely taken to mitigate licensing issues, and to work around the high level of technical skill necessary to work with large-scale data.

This thesis takes the view that representation is largely a solved problem: UTF-8 and XML are both capable of storing international characters and complex annotations in a way that is easily mined for many uses, and advances in database technology and distributed processing offer many ways to process and retrieve structured data.




\subsubsection{Classification}
Efforts have been made to address the ambiguity of some often used strata such as text-type and genre.  One of the more notable was EAGLES\cite{EagTcwgCtypeaglespreliminary}, which produced a number of recommendations designed to be applied to new corpora whilst retaining general compatibility with existing designs.


To some extent, the form a corpus takes is defined by its content---any meaningful taxonomy is arbitrary and theory-laden.  For this reason, recommendations made by EAGLES were driven by a review of the theoretical basis for existing taxonomies and work at the time\cite{EagTcwgCtypeaglespreliminary}\td{page unknown (online)}:

\begin{quote}
Any recommendation for harmonisation must take into account the needs and nature of the different major contemporary theories.
\end{quote}

The need to provide useful metadata is particularly challenging for general-purpose corpora, which have very loosely-defined aims and must maintain a high level of neutrality.



%--

Internal text distinctions may be drawn using bottom-up (often termed `corpus-driven') methods, however, these are frequently difficult to operationalise.  A prime example of this is Biber's multidimensional analyses of corpora\cite{biber1992complexity}, which focus on feature extraction and principal component analysis in order to identify primary dimensions of variation within corpora.

Biber's approach has been held aloft by many as one of the only `unbiased' analyses of variation around, however, this is not the case.  Without an authoritative underlying theory of language use by humans, any features used to construct the model will, however neutrally treated thereafter, exert pressure on the results.  Given the subtlety with which such analyses extract information (and the difficulty in interpreting resultant factors), this is likely to lead to favouring one set of conclusions over another.

This is also true of higher-level techniques such as latent dirichlet allocation (LDA), which is usually trained using token frequencies\cite{Blei2012PTM1338062133826}.  These straight frequencies are still entirely bereft of context, and tokenised using procedures that embody particular theoretical decisions such as the importance of punctuation\cite{pentheroudakis2006tokenizer} (especially true for languages lacking word delimiters, such as Chinese\cite{Wu1994ICT974358974399}).  

Sharoff\cite{sharoff2007classifying} presents a more transparent clustering methodology that relies on keyword metrics to identify salient features.  This provides some connection to other keyword literature, along with a mechanism for inspecting the resultant categories.

Arguably, further problems with the approach of finding natural strata of variation lie with sampling and linguistic problems: many of the practical issues surrounding corpus building involve enumeration of easily identifiable groups of texts, something that would be hard to compromise or `trade off' if working from factor loadings.  Further, many analyses will find subsampling data difficult when defined in terms of many external variables\cite{aston2001text}, though this is more of a challenge for the linguistic community (and providers of its tools).

%--

Through reasoned examination of existing efforts, Lee\cite{lee2003bnc}
was able to re-form the BNC's classification by adding external data and classifying documents manually.  This is arguably the most prominent effort to apply the multi-phase `examination and re-appropriation' method that Biber and EAGLES recommend.

His methodology, though labour-intensive, offers a defensible way to trade off the various interests of users.  One key aspect to this is the fact that it was built after the corpus, and thus may take into account common usage when making distinctions between texts---something that Lee relies on in an effort to define categories using `external' definitions (as opposed to Biber's internal variance measures).  This may prove more easy to operationalise in some circumstances, but still involves the subjective expert opinion of someone who may or may not agree with the user's perspective\cite{aston2001text}.


In summary, the problem of producing a meaningful and operationalisable taxonomy for corpus organisation is actually one of community agreement: for any given user of a corpus, the task is simpler.  This indicates that a transparent, well-documented approach should be taken in order to allow end users to decide upon their level of agreement with the pre-applied categories, or that differing levels of confidence should be applied to aid subsampling.








\subsubsection{Dissemination and Collaboration}
The ability for multiple researchers to access a corpus is one of the main benefits of corpus methods---corpus-based studies may be replicated and compared with absolute certainty of the empirical aspect of the research.

The process of building a large, multi-purpose sample for use by a whole field of research mirrors that used in other fields, many of which have similar problems gathering data.

Many examples of this approach exist, such as the British Household Panel Survey and British Crime Survey\cite{taylor1996british,hough1983british}, both of which demand[ed] significant investment over a long period in order to overcome the practical issues of large-scale demographic sampling.  Nonetheless, the quality this yields has lead to their widespread use, for example, both are used extensively by governments to assess social policy impact.

Although corpora are generally less sensitive on ethical grounds, many legal challenges remain to distributing and using such a large quantity of data.  This has historically greatly limited both the source and form of data gathered for corpora, and was one of the reasons behind sampling 2000-word samples (rather than whole texts) in the Brown corpus.

Navigating copyright law remains one of the primary tasks of a corpus building effort, though the increasing dependence on digital sources has dulled this somewhat, since they are typically less controlled when being published.  Nonetheless, corpora often come bundled with restrictive licensing, something that limits the ability of the wider community to participate in their use.

Many countries' fair use exemptions apply to research, though this may apply only to copies taken for private study, as laid out in the UK's Copyright, Designs and Patents Act 1988
%citation: http://www.legislation.gov.uk/ukpga/1988/48/introduction
.  The concept of `Fair Dealing' covers many uses of extracts in research, and a recent exemption was added that additionally allows the use of\cite{intellectualpropertyofficeuk2014}\td{nopage, online}:

\begin{quote}
\ldots{}automated analytical techniques to analyse text and data for patterns, trends and other useful information.
\end{quote}

% Similar fair use clauses are relied upon in the USA to distribute some web corpora, 
%-



The heavy re-use of corpora, then, exerts both positive and negative forces upon the scientific community.  On one hand, it is necessary to share resources in order to lower costs, increase awareness of rare data, and pool efforts to create better samples.  On the other, the ubiquity of a corpus may damage its scientific value, and starve the community of more up-to-date or relevant resources.  In the worst case, widespread use of a single corpus by the community may lead to partial circularity of hypothesis derivation and testing.

% \til{I don't know enough linguistics to pick out features that have been hot topics enough to do this.\\
% Also I fear this is hard to find due to the lack of distinct exploratory/confirmatory roles in CL papers.}

In an ideal world, it would be possible to replicate studies with one's own samples.  As I shall cover throughout this thesis, increasing digitisation of documents (and use of the web) offers a way to make this scenario possible, at least for some corpora and purposes.



\subsubsection{Sample Type}
Corpora span many types of sample, from simple cross-sectional ones (`synchronic' corpora: Brown, BNC) to those that aim to report language use through time (`diachronic' corpora: ICE, Longman/Lancaster) and a hybrid of the two, which is designed to follow language use and update on-the-fly (`monitor' corpora: BoE, COCA)\cite{francis1961brown,burnard1995users,greenbaum1996international,summers1993longman,Jarvinen1994AMW991886.991985,davies2010corpus}.  Further to this, there are parallel corpora, intended to match texts across some variable such as language\cite{koehn2005europarl,mcenery2007parallel}, and many other designs that combine properties of these to satisfy various sample designs.

These approaches represent various use-cases, and are often significantly less `general-purpose' than large synchronic reference corpora such as the BNC.  This thesis treats the selection of a sample design as problem-dependent.








\subsubsection{A working Definition}
The above discussion illustrates the wide variety of samples that may be called general-purpose corpora, and some of the issues that affect their utility.  The focus of this thesis is on mechanisms for making, operationalising and documenting the above choices, and as such the working definition here opts to not apply any clear restrictions.

One obvious requirement, however, is that a corpus must be a suitable sample \textsl{for its intended purpose}.  This means that the sample design (external variables) and the document classification (internal variables) must be clearly defined in terms of the research questions given.

For the purposes of this thesis, a corpus constitutes a body of text that must be:

\begin{itemize}
    \item Representative of some stated population;
    \item Sufficiently large to satisfy that representation (for a given set of purposes);
    \item Explicit in its coverage of external variables;
    \item Of a regular, machine-readable format.
\end{itemize}


% 
% 
% % TODO: some kind of blockquote thing
% ``A collection of text samples, subject to the following conditions:
% 
% \begin{enumerate}
%     \item \label{enum:corpusdef-external} External (contextual) data is sufficiently well documented to prove useful to research (i.e. the population is well defined, texts are annotated);
%     \item \label{enum:corpusdef-internal} Texts are sampled in an internally-consistent manner, relative to external proportions (i.e. the corpus is externally and internally valid, and thus representative);
%     \item The data are recorded in a format suitable for automated processing (i.e. the corpus is not intended for direct manual inspection, and will be sized accordingly);
%     \item Documentation is provided in order to describe the veracity of claims made for \ref{enum:corpusdef-external} and \ref{enum:corpusdef-internal} above.
% \end{enumerate}
% 
% ''
% 
















% Here there is a conceptual shift from "Current practice//problem" to "Ideal practice//solution"
\section{A Formal Perspective}
\label{sec:litreview:sampling}
% This will be a review of more formal sampling theory, comparing it to methods for acquiring language.


Corpus building methods are largely based on sampling methods from the social sciences.  These methods are well developed, and their formal frameworks specify a number of design choices that must be made whilst designing a sample.  
These decisions largely affect the suitability of a corpus for different forms of analysis, and the frameworks they are based on may be used to motivate design of the sampling process itself.
Re-examining the original principles of sample selection allows us to use some of the formalisms developed elsewhere to inform judgements on the properties of linguistic samples.

The goal of any sample is to present a scaled-down set, containing individuals that represent all variation within the population.  Before discussing the implication of various approaches, it's therefore important to draw a distinction between variables which are controlled by an experimenter (independent) and those that remain free to vary (dependent).

This distinction has a large impact on sample design, as it is impossible to draw conclusions about the population by inspecting independent variables.  Further, the selection of documents according to these controls often leads to systematic bias in dependent variables.  In order to come to conclusions about representativeness and sample quality, it is thus necessary to identify these variables ahead of time.
% It is this that leads me to stress the importance of corpus documentation

In the case of sampling documents, I will be presuming that users of corpora are primarily interested in inspecting `internal' text features, which are described in terms of `external' document metadata.  This guiding principle mirrors the metadata/data dichotomy seen in corpus tools, and should be uncontroversial\footnote{Note that many definitions of genre, text type, etc.\ make this circular, as they are defined by document content.}.  In this thesis, I will often label variables as internal or external based on these criteria, with the implication that external variables are independent, and vice-versa.

The ideal sampling scheme for a given population and selection of variables, then, contains variables as controlled specifically for the research question about which one wishes to infer, and maximises coverage of internal document content (and uncontrolled-for metadata values).

Taking this principle to extremes yields the maximally-representative census sample: 100\% of the population of interest.  At this point, any inference is mere observation, and the only potential pitfalls are ones related to whether or not the question itself is worth asking, rather than the validity of its answer.  A census still contains theory-laden assertions, however, in the form of its population definition\cite{atkins1992corpus}.


At the high level, samples may be classified into two main groups: \textsl{nonprobability}, and \textsl{probability} samples.  The former of these is primarily guided by a systematic or subjective choice, and the latter has a sampling frame defined by random selection.

% For statistical analysis, probability sampling is necessary, with the simplest case being simple random sampling (SRS).  In practice such a thing is seldom possible, and methods such as weighting and stratification may be used.
\subsection{Nonprobability Sampling}
The selection of data points in a nonprobability sample is performed either by an objective system, subjective reasoning, or some combination of the two.  Factors influencing selection are often situational or theoretical, meaning that samples require a greater understanding of the subject to avoid accidental biases.  

% Because a number of  there are often nonprobability elements that creep into larger samples.

Three main forms of nonprobability sampling are identified by Barnett\cite{barnett1991sample} and Teddlie\cite{Teddlie01012007}: \textsl{availability sampling}, \textsl{purposive sampling}, and \textsl{quota sampling}.

\subsubsection{Availability Sampling}
Also known as convenience sampling, this approach simply takes the most readily available data points.

This method has been widely used in the social sciences, with experimenters often using students from their affiliated institutions (an approach used for some very famous studies).

Convenience sampling is rightly regarded as extremely unrepresentative of wider populations, particularly in fields (such as psychology) where there are a great number of covariates.  If corpus linguists were to work with data sampled as conveniently as those in the Stanford prison experiment\footnote{``The participants were respondents to a newspaper advertisement, which asked for male volunteers to participate in a psychological study of ‘prison life’ in return for payment of \$15 per day.''}\cite{zimbardo1971stanford}, they would merely be reading books from their own bookshelves.

There are a number of methods that are designed to adjust for these biases.  Birnbaum \& Munroe\cite{birnbaum1950munroe} present a model of the bias resulting from unavailable population members after random selection.  This approach is difficult to operationalise in a linguistic context as they require enumeration of data points prior to determining whether or not they are available, rather than the more haphazard approach of true convenience sampling.

Farrokhi\cite{farrokhi2012rethinking} approaches the problem of constructing two groups: a control and treatment group, using a series of predetermined criteria to assign membership.  This approach mirrors the comparative nature of many corpus linguistic methods, but does not directly offer a way to produce a more representative corpus for `general use' aside from the principles of using heuristics to guide design.

Despite the drawbacks of conveinence sampling, such an approach may be appropriate for populations where it is agreed that there is little variation between individuals for the variables of interest: for example, a study which seeks to establish the modal number of eyes humans have might be well served with a very small sample.  Smaller linguistic features such as unigram frequencies are likely to be similarly easy to represent.



\subsubsection{Purposive Sampling}
Purposive sampling describes the case where those constructing the sample make a deliberate and systematic choice of inclusion, based on expert opinion.

This approach is primarily effective against well-known sources of bias, and the validity of any sample built using it is entirely dependent on identification of these.  Where the expert design encompasses all variables of interest to a study, and where selection has been performed in such a way as to encompass individual data points from across the population, such an approach may result in a high-quality and defensible data set.

The main difficulty is in avoiding previously-unknown correlations between the theoretical selection criteria and the study variables.  As selection criteria are based on domain knowledge, and heavily theory-laden, it is often difficult to anticipate their interactions with other variables of interest.  As theories improve over time, this may also to lead to the effect of samples being considered more biased as knowledge of an area improves, gradually reducing the perceived validity of any results.

Criteria for selection generally accomplish differing goals, and will suit varying study aims\cite{advice2000study} (for example, some of these may be very useful for exploring rare features):

\begin{itemizeTitle}
    \item[Heterogeneous] An attempt to cover as much of the population as possible, by selecting data points that are unlike ones already in the sample.
    \item[Homogenous] Data points are selected to be as similar as possible according to given variables.  This is suited to in-depth qualitative analysis, being somewhat analogous to merely controlling for more variables (i.e.\ reducing the population).
    \item[Typical Case] Data points are selected according to a theoretical/rational idea of typicality.
        %This is distinct from statistical representativeness (which relies on the central limit theorem).
    \item[Extreme Case] Only those data points regarded as atypical are sampled.  May be used to explore reasons for atypicality.
    \item[Critical Case] Data points are selected based on previous theories, such that they explain certain hypotheses.
    \item[Total Population] An entire sub-population is sampled, due to its ease of access (for example, all members of an organisation, or all publications by a certain author).  If no other items are sampled, this merely becomes a census with a very tight population.
    \item[Expert] Expert opinion is used to determine inclusion in the sample.
\end{itemizeTitle}

From a corpus construction perspective, where the sampling body is often distinct from the analyst, the complexity of sample inclusion criteria brings with it the need for extensive documentation: without awareness of the expert's choices, it becomes very difficult to defend any use of the sample.  Due to the nuanced nature of their validity, purposive samples are valid only for qualitative analyses, where interactions with the study design are able to be rationalised and explained in context.

This is especially the case where a study's purposive design is borrowed for use in other contexts: for example, the Brown corpus explicitly states its aim to represent `standard' English, yet its sample design has been widely copied\cite{hundt1999manual,shastri1988kolhapur,mcenery2004lancaster}.

In fields which lack a cohesive, quantitative model of their population, it is often necessary to start from an expert-defined base.  Ideally, information from this sample may then yield methodological improvements, and, eventually, an unbiased mechanism for retrieving a statistically representative sample.
% cite: http://dissertation.laerd.com/purposive-sampling.php


\subsubsection{Quota Sampling}
This is a two-stage design, with a number of sub-populations being identified and then selected on a purposive/availability basis.  It is essentially a nonprobabilistic form of stratified sampling: the population is split into mutually exclusive groups, into which data points are placed until each group is `full'.

This approach is often used to control convenience sampling variation for some important variables, for example, controlling for sex and age whilst performing market research.  The BNC's spoken portion was `balanced' using this method, with a number of bins being allocated by context and speaker information.  The COCA corpus was also constructed using this method using data from the web~\cite[p. 163]{Davies20090601T0000001384-6655159}:

\begin{quote}
Using VB.NET (a programming interface and language), we created a script that would check our database to see what sources to query (a particular magazine, academic journal, newspaper, TV transcript, etc.) and how many words we needed from that source for a given year.
\end{quote}



\subsubsection{Summary of Nonprobability Methods}
Corpus sampling methods described above already closely resemble quota sampling, using a lot of expert opinion to define the quotas.  Nonprobability sampling, however, is particularly ill suited to statistical analysis and inference, which relies on random selection over uncontrolled variables.  Simply, nonprobability sampling techniques are very easy to bias~\cite[p.19]{barnett1991sample}:

\begin{quote}
\ldots{}there is no yardstick against which to measure `representativeness' or to assess the propriety or accuracy of estimators based on such a sampling principle.
\end{quote}

This opacity leads to limitations in corpus analysis, where the goal is to use large volumes of data as objectively as possible.  For any quantitative analysis to be scientifically defensible on empirical (rather than rational) grounds, probability sampling is \textsl{required}.




\subsection{Probability Sampling}
There are many probability-based designs available, and the choice of them is largely dependent on the methods of inquiry that are to be applied to the resulting data.  In all cases, the goal is to allow the dependent variables to vary randomly, ultimately allowing for statistical inference if sample sizes are sufficient.

Random selection of data points provides both a mechanism for unbiased estimation of parameters such as means, as well as knowledge of variation.  The latter of these is the key difference between a probability-based sample and a well-chosen nonprobability one, and is vital to basic hypothesis testing procedures.  Further, notions of sample size are entirely based on these measures: power analysis demands some knowledge of variance and effect size, both of which are only meaningful under conditions of random selection.

Random sampling methods identified by Lohr, Barnett, and Teddlie include\cite{lohr2009sampling,barnett1991sample,Teddlie01012007}: \textsl{simple random sampling}, \textsl{stratified sampling}, and \textsl{multi-stage sampling}.

\subsubsection{Simple Random Sampling}
This approach selects members of a population entirely at random.  Each member of the population has a probability of selection that is simply $\frac{sample~size}{population~size}$.

SRS is free of any errors that stem from classification, reducing the importance of potentially circular genre definitions and taxonomies.

The main disadvantage is that it requires a complete sampling frame: that is, all individuals in the population must be known in order to be randomly selected from.

Though corpora often contain randomly sampled portions, for example, where data is enumerated by a publisher's list or online directory, the lack of a central authoritative index that covers the whole population is usually an impediment to retrieval of a truly representative sample.  Essentially, simple random sampling is unsuitable for larger samples in linguistics due to this limitation.

Once a corpus has been constructed, it is often possible to use SRS in subsampling approaches.  Its unbiased properties also make it useful in bootstrapping, allowing methods to use the full distributional information contained within the corpus\cite{gries2006exploring}.  These techniques serve to avoid introduction of further bias, but don't sidestep any representativeness issues with the original corpus.

% \til{examples from CL?}

\subsubsection{Stratified Sampling}
Stratification is the process of breaking a random sample into a set of bins, with sizes weighted according to some policy.  In the case that the strata are selected in an unbiased manner, this should yield the same sample as SRS.

This method ensures that at least one individual is selected from each stratum: something that may be used to improve representation for populations with highly heterogeneous distributions.  This adjustment can be tweaked to allow comparison of low-incidence and high-incidence effects by deliberately increasing the proportion of the population sampled for infrequent strata (for example, selecting more people from areas with low population density).

Stratum selection should be along real-world subpopulation boundaries, and is usually selected in order to maximise representativeness (by ensuring that strata are sized according to the population) or statistical power (by ensuring that strata are sized according to the amount of variance in the strata).

All stratification requires the ability to exhaustively separate the population into mutually exclusive categories, and sample from these categories in a random manner.  Both of these are a challenge in corpus linguistics, as removing the first level of `no indexing' often leads to another.  Nonetheless, where information silos exist (such as in academic publishing) then this approach is particularly suited.

Where auxiliary data exists, statistical benchmarking may be performed by adjusting sample weights according to the proportions seen in this data.  This may be applied after sampling itself (otherwise said auxiliary data is simply used to determine the initial strata sizes), and so is often referred to as `post-stratification'.  This approach is applicable where ordinary stratification is impossible, for example, where the variables on which to stratify are unknowable at the time of sampling.


\subsubsection{Multi-stage Sampling}
Multi-stage sampling involves multiple rounds of random sampling with progressively diminishing sampling units.

Multi-stage sampling is particularly applicable in cases where \textsl{all} of the data points in the population are accessible through a hierarchical structure.  This is clearly the case for smaller linguistic features, but much less so for whole documents (or large extracts).

This approach often significantly speeds up the process of enumerating and selecting from a population, and reduces the need to classify things compared to stratified approaches.  The increased structure may also be of use to some models, allowing for multi-level modelling approaches during analysis.  Because the method relies on excluding large areas at once, it is less representative than SRS for the same sample size, and this may complicate analysis.

Cluster sampling, a form of multi-stage sampling that relies on existing organisation of data, has particular applicability to web sampling, where documents are often stored in academic repositories or simply under the same domain, or to sampling from different publishing houses.

Evert provides a library metaphor for sampling in which he explains random selection of individual words using a multi-stage approach\cite{evert2006random}, selecting progressively smaller units at random before returning a single word and repeating the process.  Though he focuses on randomness, it would be possible to use this approach with stratification at each level---this mechanism is well suited to situations where auxiliary data may be available, but not at the level of the desired sampling unit.
% http://www.tandfonline.com/doi/pdf/10.2167/eri421.0

\subsubsection{Probability Sampling Summary}
Simple random sampling is often seen as the ideal probability sampling design, in that it yields high quality results using statistical methods, without the need for weighting and adjustment of results.  It is also by far the hardest method to apply to text due to its incompatibility with the process of accessing the data.

Stratified sampling is arguably simpler, in that it allows for testing and definition of strata prior to retrieval of texts, and stratum sizes may be computed from existing corpora in a multi-stage design.  Another benefit of stratified approaches is the ability to examine and adjust the distribution of strata according to expert opinion, offering a hybrid design that is able to compensate for known practical issues.  This is often used to artificially boost the stratum sizes of minor groups within the population in order to ensure that they are over-represented in the final sample --- something that may be desirable if their influence is particularly important to a research question.



\subsection{Sample Size Estimation}
Quantitative sample size estimation is largely based on ensuring that a given hypothesis test can detect an effect of interest.  This varies according to a number of parameters, including the type of test used.  Any hypothesis test is defined by four parameters\cite{ellis2010essential}:

\begin{itemizeTitle}
    \item[Probability of Type I Error ($\alpha$)] The probability of rejecting the null hypothesis when no effect exists in the population.  Forms the threshold for the oft-quoted `p-value'.
    \item[Power ($1 - \beta$)] The probability of rejecting the null hypothesis when an effect exists in the population.
    \item[Effect Size ($d$)] The observed change in a parameter necessary to identify an effect with probability $1 - \beta$.
    \item[Sample Size ($n$)] The number of individual units in the sample that are free to vary.
\end{itemizeTitle}

Sample size must be estimated with knowledge of the power of the test to be done, and of the population distribution.  A simple binomial test may have its sample size estimated thus:

$$
n = \frac{Z^2p(1-p)}{d^2}
$$

Where $Z$ is selected according to the area under the population distribution according to $\alpha$, and $p$ is the parameter value within the population (often estimated as $\hat{p}$).  This implies that a reduction in any effect size we wish to detect increases the required sample size (and vice versa), as do increases in the confidence at which we seek to reject $H_0$ at.  This general trend applies to all sample types above.

In addition to sample size, the power of a test is also affected by the method used to perform the hypothesis test.  Tests which are excessively conservative further increase the probability of missing an effect, lowering power, as does violation of assumptions upon which parameter estimates are based.

Current methods for computing sample size are reliant on assumptions of independence, often phrased such that members of the sample must be `independent and identically distributed' (IID).  This assumption is violated massively by conventional corpus construction: corpora are sampled at an extract or document level, yet often analysed at a word or phrase level.

In the absence of IID samples, sample size calculations are reduced to what Cohen~\cite[p. 145]{cohen1977statistical} termed ``non-rational bases'' such as past experience or rules-of-thumb.  The process of building a corpus that is IID for a given research question is so theory-laden as to be inapplicable to those building general-purpose corpora using the current model (where many users all share a large corpus for diverse tasks).

The reliance of corpus linguistics (and to some degree NLP) on qualitative methods and rational definitions of parameters such as population and sampling frame make application of existing power analysis to corpus construction challenging.

An alternative method, suited to qualitative and mixed-methods approaches (and thus less reliant on the randomness assumptions violated by document-level sampling) is presented by Glaser\cite{glaser1965qualitative}, who suggests that those performing experiments should be guided by `saturation': the point at which no (or negligible) new information is presented by adding new data points to the sample.  This mirrors the methods of Good-Turing frequency estimation\cite{GOOD01121953}, as used by Biber in his 1993 assessment of representativeness\cite{biber1993representativeness}.

Saturation-based measures offer a less-formal mechanism for assessing the adequacy of sample sizes, however, they still require definition of a study design prior to assessment of sample size sufficiency.  Such methods are also particularly lacking when sampling is nonrandom, as they rely on the discovery rate of new information being unbiased---if retrieval of documents is systematic the rate of new information will plateau, leading to premature conclusions of coverage.  Again, without the ability to enumerate the population, it is difficult to insure against such bias.




\subsection{Sample Design}
The design of a sample can be broken into a number of stages, each of which informs the next.  As we have seen, many of these require theoretical justification or analysis based on the final study design.  This forms the framework used elsewhere in the thesis as an ideal case.

% \til{This isn't cited because it's novel --- I consider it a summary.    }
% http://blog.reseapro.com/2012/11/sample-designing-steps/
% http://korbedpsych.com/R06Sample.html
%

\subsubsection{Research Question Definition}
The definition of a research question is an important first stage to selecting a sample.  This is in order to define any theoretical assumptions and the analysis design, both of which have significant impact on the size and form of any sample.

\subsubsection{Sample Unit Specification}
The unit of analysis should be in agreement with the aims of the research question above, and ideally should match the sampling unit.  Where disagreement occurs, this is likely to abrogate the quality of any analysis based on IID assumptions, including power analysis for sample size estimation below.

\subsubsection{Variable Selection}
Dependent and independent variables should be specified unambiguously, and any relationships between the two should be examined from past experience and existing literature, in order to reduce unanticipated correlations causing bias.  Where the research question and intended analysis are already rigorously designed, this task should be fairly transparent to any circularity, however, this becomes more challenging as qualitative elements are included.

\subsubsection{Population Definition}
The bounds of the sampling frame must be defined in terms of the variables to be sampled, such that any retrieval efforts can unambiguously use these.  A population definition is, in effect, definition of an independent variable which is controlled prior to sampling, and as such its relationship to dependent variables and to the original research question is similar to that of any other variable.

\subsubsection{Sample Policy}
The manner of sample must be decided based on practical issues, and on the various properties specified above.

\subsubsection{Sample Size}
Depending on the design and analysis method proposed, this may take the form of a quantitative power analysis, or a plan to implement qualitative controls during construction of the sample.  Some a-priori objective criteria should be defined, particularly when a sample is based on qualitative assessment, in order to avoid data dredging (or `dredging' of variables correlated with those to be studied).

\paragraph{}
The existence of previous studies, literature, and datasets may guide all of the above stages, and the progression from expert-opinion-based sampling to a more quantitative understanding of population variance is made possible by iterating the above: something that happens naturally as a field progresses.

It is notable that traditional general-purpose corpus construction efforts are largely unable to specify research questions ahead of time as they are necessarily retrieving data prior to the study even being conceived.  It is also the case that most studies include a significant qualitative component, in interpreting the output of quantitative models or in the form of direct reading\cite{rayson2008keysem}.
It seems likely that these properties pose the biggest challenge to the classic question ``how large should a corpus be?'', which requires more rigorous specification of these.






\subsection{Sources of Error in Corpus Construction}
The validity of a sample is based not only on its representativeness for a given question, but also how well that question may be related to reality, and what limitations are imposed by the process of retrieving data.
Corpus builders are not blind to these challenges---indeed, most literature on corpus construction devotes a large portion of its content to practical issues\cite{wynne2005developing,atkins1992corpus,EagTcwgCtypeaglespreliminary}.

% \til{from bartlett book}


The requirements discussed above are satisfiable in a number of ways, each of which will exclude and promote certain uses of the result.  Generally, threats to the validity of inquiry based upon these samples may be broken down into three areas:

\begin{itemizeTitle}
    \item[External Validity] How relevant the sample is to the population about which we wish to infer something.  These are likely to limit the generality of a study, or lead to under/overestimation of its effects.
    % (Representativeness vs. transferrability, coverage of population, size [sampling unit]) sampling design issues
    \item[Internal Validity] How much a study can rely on document annotations and data in order to draw conclusions.  Issues here are likely to cause false results.
    % (internal proportions/categorisation, stratification, covariates, comparability)
    \item[Practical Issues] Limitations on the mechanics of sampling.  These issues may cause either or both of the above.
    % that prevent gathering an ideal sample
\end{itemizeTitle}

This section identifies potential issues with corpus sampling methods.  The majority of these are practical issues for which fixes must be designed carefully and on a case-by-case basis.


\subsubsection{Distributional Issues}
This category largely describes the manner in which certain important covariates are selected for sampling, or sampled.  Language contains an unknown (but undoubtedly very large) number of possible dimensions to study and compare, and selection of features to describe variation across texts is far from simple.  Such covariates may be listed as external metadata (author age, genre) or for linguistic features (prepositions, wh-relative passives).

Atkins et al.\cite{atkins1992corpus} provide a series of recommendations for covariate selection, which has been used to guide this discussion.

The population for whom a corpus describes language use is relatively difficult to describe, as it is largely a self-referential problem.

Seemingly, one reason why corpora are not demographically representative is because their selection processes rely on lists that are compiled using data for which it is difficult to determine comparable bounds.  The BNC's policy for selecting written materials, for example, is segmented into the following sources:

% \til{Insert more detailed overview, more critical analysis thereon}
\begin{itemize}
 \item Books, selected from bestsellers, literary prices, library loans, additional texts;
 \item Periodicals and magazines (including newspapers);
 \item Other media (essays, speeches, letters, internal memoranda).
\end{itemize}

Of the lists used in the books category, they specify the following criteria~\cite[p. 9]{burnard1995users}:

\begin{quote}
Each text randomly chosen was accepted only if it fulfilled certain criteria: it had to be published by a British publisher, contain sufficient pages of text to make its incorporation worthwhile, consist mainly of written text, fall within the designated time limits, and cost less than a set price.
\end{quote}

It is often difficult, using other sources of data, to establish the details of an author's nationality, the age of a text, or the suitability of the time limits\cite{dollinger2006oh}.
% Indeed, it seems likely that these bounds are likely to vary according to which type of text is sampled, but this is a relatively minor issue which would complicate use of the corpus).
These issues are typically better addressed by more specialist corpora, which are subject to easier-to-determine bounds such as social role, context or text type\cite{kucera2002czech,przepiorkowski2008towards,kyto1993manual}.

Ambiguity in population definition has a number of direct influences on other issues of corpus validity. Selection of stratum sizes, for example, must be based on the relative proportions of language used by the given population.  If a population is defined using purely internal (linguistic) properties, this becomes a reflexive and circular task---we end up selecting proportions of language to match proportions seen in language.  Use of auxiliary data to augment sampling policies (such as social demographics taken from other, large-scale surveys) must also be matched to the population in question.

Where selection of genres is defined by linguistic content, it is necessary to ensure that the frequencies used do not have systematic correlations with features being studied.  This is impossible to automate, and must be assumed based on experience and theoretical reasoning.

%-- 

Speech corpora are often specified in a significantly less ambiguous manner (though not always with more proportional sampling). This seems to be due to the direct nature of speech sampling, which involves the person actually performing an utterance (rather than the language use itself being a persistent and concrete artefact).

This difference is identified by Leech\cite{leech2006new}, and will be inspected in greater detail below, as it is a potential major source of disagreement between construction and use. % [section about production/consumption]\td{comment more}



One source of ambiguity when defining a population is inherent `measurement error' in the classification scheme used to delineate the texts.
Efforts such as EAGLES\cite{EagTcwgCtypeaglespreliminary} identified this issue as a significant one, and much effort has since been spent on minimising the problem of classifying texts.  Nonetheless, the difficulty in identifying a widely-agreed-upon classification scheme means that this remains an important design decision.

This manifests in two ways. The former being the problem of selecting texts according to external variables about which we may wish to infer findings, such as level of education of the author, nationality, target audience of text, etc.  Many of these factors are imposed by existing structures and systems of classification which were developed for some other use (e.g.\ search engines, library lists).  The latter of these issues is selection and stratification according to externally-imposed, but textual features such as genre, text type, medium, etc.  These two issues are likely to overlap somewhat: those distinctions useful for one research question may prove ambiguous for another.

A secondary, yet related, issue is that of metadata completeness: often, texts are sourced from very different places, and come with differing levels of metadata.  Normalising and homogenising these metadata is particularly challenging, and may include a loss of resolution.

The temporal aspects of texts are a special case of this.  Diachronic corpora seek to represent a `slice' through time of language use, and this requires a representative distribution of text age across the stated population.  Much of this distribution is defined by the production/consumption balance chosen for the corpus: for a given population, how old are texts used on a day-to-day basis?

When sampling online, or sampling from historical texts, merely identifying the age of a text is also challenging.

These questions are also important when analysing historical or monitor corpora, where they must be answered in order to allow accurate sub-sampling.  Corpora such as EEBO\cite{blum2002early} include such wide temporal coverage that their use often requires identification of a historical context.







\subsubsection{Practical Issues}
Copyright is one major problem with sampling large volumes of text from any source, especially those already available for-profit from large publishers, who are acutely aware that their entire business model is based on controlling access to their intellectual property.

This is stated in the original documentation for the Brown corpus as one reason for their choice of sampling unit, and often causes `black spots' in the sampling frame of a corpus, where certain publishers are known for their absolute declination in offering material.

\til{Tony's talk at CL'11 had this mentioned but his abstract in the proceedings doesn't.\\
Not quite sure which bits of BYU are worth mentioning---resale of data?  I think much of this topic is beyond the scope of the thesis but it might be worth a mention}
% http://www.birmingham.ac.uk/documents/college-artslaw/corpus/conference-archives/2011/abs-5.pdf
%

As a large number of documents are designed as ephemera, these are often not to be found in archives.  As a result, out-of-date ephemera are difficult to obtain retrospectively.  Many digital ephemera are targeted at individuals, introducing ethical issues analogous to covert research.

For speech data, and written data that is not intended for public dissemination, privacy must be considered as a legal and ethical issue.  Any auxiliary data is unlikely to suggest proportions for documents used within a private context (e.g.\ notes left on a fridge), or those controlled by formal social structures (such as documents used in many offices).

Some topics and contexts that ought to be represented are difficult to sample due to the cultural taboo surrounding them. This is especially relevant for speech corpora, since speech is often used for informal transactions, and because it is difficult to separate the sampling procedure from the speaker himself.

These issues are particularly pertinent to verbatim recordings in natural settings, such as those used in many spoken corpora.  Such recordings, even if taken in public places, may be subject to laws such as the Human Rights Act 1998 which, through guarantees of privacy, may restrict the release of such data to those not present at the time.

Such ethical pressures pose a challenge to unbiased sampling, and one that cannot easily be worked around.  The issues surrounding large-scale covert research are discussed in more detail in Chapter~\ref{sec:personal}, Section~\ref{sec:personal:discussion:ethics}.

Finally, though this is increasingly simple as the world moves towards digital storage, transcription and physical acquisition of data is often slow, difficult, and (thus) expensive.

Paper documents must be scanned, digitised, manually corrected and converted into a normal form. Speech must be gathered in an ethically defensible manner, and metadata about the speakers must be gathered.  Electronic documents require significant format conversion.

These problems are easing: digitisation technology, especially that used for paper documents, has recently progressed significantly due to efforts to preserve historical documents, and libraries' attempts to digitise older publications. In some cases, scanners are capable of scanning entire books without intervention.








\subsubsection{Stratification}
Though a problem for categorisation and taxonomy selection, the `fuzzy edges' of genre, medium, popularity and other covariates often leads to ambiguity surrounding which category a text should belong to.

Some classification schemes apply multiple labels to a text in order to avoid this\cite{sharoffs2015}.  This approach may provide a way of producing a more accurate overall classification, but complicates many analyses, particularly quantitative ones relying on regression.

Biber and others\cite{leech2006new,biber1993representativeness} recommend that sampling should occur in an iterative process, with the contents of a corpus being used as evidence to weight selection of strata from the next version.

As Varadi\cite{varadi2000corpus,varadi2001linguistic} notes, this simply doesn't happen.  Reasons for this may include the shift in classification priorities, and the time required to re-code and align a previous corpus' annotation structure to that which is to be built. 

% Having acquired a first iteration of a corpus, there are many methods for assessing the 'completeness' of a sample.  some of these are explored by Biber, and others (including Evert, Greis) go on to describe further internal measures\cite{evert2004simple}.

% Note that this is quite aside from the problem of assessing corpus proportions using external variables, which is easier to inform using other studies in a multi-phase fashion (using demographic proportions, for example).

Auxiliary data from social surveys may be used as a rough indicator for this, though in reality no data source exists that can describe, in social terms, the `types' of language used (w.r.t.\ the primary dimensions of sampling for corpora).  Questionnaires, ethnography, and/or direct sampling may be the only ways to establish a ground truth for this, but for now it remains an open research question.% [until the personal corpus section ;-)]





% This problem can be seen two ways, depending on the sampling policy.  In an ideal world, strata are entirely proportional, and the problem of identifying the minimum size is one of ensuring that all variability within the smallest stratum is adequately represented.  In practice, one of the reasons for stratified sampling is to, with philosophical rather than mathematical justification, boost the prevalence of a stratum by artificially oversampling it.

% Both approaches are correct from their respective viewpoint, however, the deliberate oversampling seen in corpus construction is often informal.  Algorithms for internal saturation/sufficient sample size may be used for both the formar and latter, though it's worth noting that the sample is less random if used in the latter way.  The highly Zipfian distribution of language will, for most practical purpose, necessitate the use of inflated stratum sizes for small strata.



\subsubsection{Randomness}
The pragmatic issues surrounding corpora do not apply equally to all genres, media, social demographics, or settings.  This results in the need to apply large qualitative corrections to the sample design, which restricts the ability to use random sample designs.

A prime example of this is the proportions of written and spoken texts in many corpora, and even the proportion of elicited spoken vs. `natural' spoken texts, due to the difficulty in obtaining consent.  Another example may be seen in the BNC's proportion of academic or newspaper texts, both of which comprise a very large proportion of the corpus, yet are read by a relatively small proportion of the population.

% (i.e. web crawls)
In many cases, such as web crawling, nonrandom sampling strategies are used due to the lack of an authoritative (or reliable) central index.  In others, such strategies may be the result of other practical issues, such as the location of researchers or legal concerns surrounding certain contexts.

It's possible to augment and correct for a lot of the problems that are introduced through nonrandom sampling, for example, Schafer and Bildhauer do this in their web-scraping corpus building tools, which attempt to stratify their samples by top-level-domain \textsl{after} selection\cite{schafer2014focused}.

Nonrandom sampling is an unavoidable truth of corpus building in almost all contexts, and with care should not unduly influence the result of a corpus building effort: after all, most fields face similar practical issues.
%With proper adjustment, this problem is to be of equal or lesser magnitude to that of omitting some variables from stratification efforts.

Sample size calculations are a complex topic, particularly where the variance for a given population is difficult to determine.  Due to limitations in power analysis methods for mixed-methods designs, many of the sample size judgements required during corpus construction will necessarily include some expert opinion.

Evert's LNRE models offer possibly the most advanced generative method for this\cite{evert2007zipfr}, and can be seen as a progression upon Biber's variation analyses of the early nineties.  The issues with using such models to determine corpus variability sit neatly in two categories:

\begin{enumerate}
 \item Selection of interesting variables to measure sufficiency (this, in reality, would be part of the corpus documentation: we can say it is sufficiently representative for frequency counts, grammatical analyses up to $n$ tokens, etc.\ with little risk)
 \item Decisions on how much data is enough data.  This is fairly easy to approximate with a sufficiently accurate language model, and even simple binomial approximations prove useful in judging the likely frequency of features.
\end{enumerate}

In spite of the practicality of the techniques surrounding measures of internal feature saturation, few currently do such analyses.  This is perhaps because the linguistic researcher himself must usually perform the analyses in terms of his own study criteria, rather than the corpus builder.

The problem of quantifying variation was tackled by Gries\cite{gries2006exploring}, where he presents a method for estimating confidence intervals for small-scale features, and uses these to explore the homogeniety of corpora and their internal distinctions.  Therein he focuses not on simply frequency variation, but on grammatical properties that, he claims, are more representative of the sort of inquiry made using corpora.  He starts by using the $Z\--\-score$ centrality measure, noting the limitation of existing partitioning schemes~\cite[p.123]{gries2006exploring}:

\begin{quote}
In order to measure the homogeneity of a corpus with respect to some parameter, it is of course necessary to have divided the corpus into parts whose similarity to each other is assessed. 
\end{quote}

He then goes on to augment this method by way of bootstrapping, in order to identify regions of maximum homogeneiety within a corpus in a ground-up fashion.  Such permutation testing offers a ground-up mechanism for identifying genres, at the potential cost of interpretability.




\subsubsection{Sampling / Analysis Unit Mismatch}
% i.e. For a given corpus size in words, the size in units may be too small.
Any sample is, strictly speaking, best analysed in terms of its sampling unit.  Any disagreement leads to a reduction in how free each data point within the sample is to vary\cite{kilgarriff2005language,lijffijt2014significance,paquot2009distinctive}.

Biber's initial assessments of language variation in 1988\cite{biber1988variation} examined the suitability of the $2,000$-word sampling unit by splitting each sample and comparing the relative frequency of features in each half.  He determined that, if they were the same, then the sampling unit size was sufficient to represent a given feature.  This lead to the conclusion that corpora were adequate for inspection of `small-scale grammatical features', something that seems to be borne out by the success of computational models in this area (POS tagging, etc.).

This issue is often phrased in terms of dispersion, i.e.\ the tendency for a feature to be represented evenly in all documents throughout the corpus.  Dispersion is something that manual inspection of concordances and corpus data renders particularly transparent, as it is often possible to see that all instances of a particular idiom are traceable to a given author, or all coverage of a certain topic is from one publication.  This is a prime example of the unit of analysis being far smaller than the sampling unit, a problem that is receiving increasing recognition.

Evert, with his library metaphor\cite{evert2006random}, describes a sampling policy for corpus linguistics that avoids this disparity.  In it, he describes randomly sampling progressively smaller units in a virtual library, moving from books through pages to sentences.%  todo: check if he stops short of words, iirc he does.

Since many statistical problems arise from the disparity between sampling and analysis units, a particularly problematic instance of this issue is the use of bag-of-words (BOW) models.  These model language without taking into account order, and as such would be best applied to a corpus sampled at the word level: any section of text beyond this is going to exhibit `clumpiness' effects that are beyond the comprehension of the model.

The prevalence of BOW models is such that many people choose to phrase their objections in terms of the accuracy of binomial models of language frequency\cite{kilgarriff2005language,evert2004simple,evert2007zipfr}.  Undoubtedly they have a point---more complex LNRE models are capable of far more useful inferences, however, it's worth noting that only a model that can integrate the linguistic influence of word 1 upon word $2,000$ in a sample will truly justify a $2,000$-word sample\footnote{This is a quantitative form of Hoey's argument that whole texts are necessary to truly describe human expectations of word use.}.

%--

One part of the problem is caused by a fixation on word frequencies. A given corpus, $100,000$ words in length, may comprise just $50$ texts. For the purposes of many analyses involving person-person variation, it can be said that we really only have $50$ data points. Though care is often taken to sample these texts broadly, the idea of targeting a corpus in terms of its word length, rather than the number of samples, leads to extremely poor suitability for analysis of many small-scale features.

The problem of quite what sample size to choose is a trade-off: if we were to sample single sentences for our $100,000$ word corpus, we would soon require thousands of samples, each from a randomly selected and carefully-stratified source. If we wished to perform a complex analysis of narrative structure within those sentences, we'd find there is insufficient data.

Except in certain cases, there is a plateau of difficulty for sample size: sampling $2,000$ words from a book incurs very similar levels of practical obstacles than does sampling $100$. Sampling units should be chosen in accordance with the complexity required by researchers of the time---for example, those working in information retrieval and NLP fields will demand relatively large sample units by comparison to many researchers in linguistics.

Corpora should, ideally, be defined in terms of the number of \textsl{datapoints}, rather than words.  This is one area where compatibility with existing corpora and techniques is arguably damaging to the final result, and where documentation should be clearer in guiding valid use.















% -----------------------------------------------------------------
\subsection{Validity Concerns in Corpus Analysis}
One of the primary reasons for using quantitative methods in research is their objectivity and empirical basis.  This is especially the case in linguistics, which seeks to generalise about a social property that is difficult to quantify or relate to other users.

In addition to the above procedural concerns, manual inspection and summarisation of corpora (or collections of features extracted by corpora) often leads to situations where errors of human judgement may imply certain findings.  These cognitive biases are widely recognised in many cases, and have been identified in other fields as common causes of error\cite{jain2012does,lieberman2009type}.

Many of these biases are fairly minor, and scientific methods are designed to counter-act their effect.  Nonetheless, qualitative analysis, poor corpus design, and presentation of certain features once extracted for inspection, can introduce their effects.


% \til{Until I decide to keep this section, here's an informal list:}

\begin{itemizeTitle}
    \item[Insensitivity to Sample Size] People are liable to underestimate variation in small samples.  Manual inspection of particularly intricate features extracted due to their grammatical form, especially from subsamples of the corpus (such as the spoken part) are likely to qualitatively imply false results\cite{rabin2000inference}.

    \item[Clustering Illusion] A tendency to find patterns where they are none (aphophenia).  This is more of an effect when seeing larger data sets, such as when inspecting frequency lists or concordances.

    \item[Prosecutor's Fallacy] Assuming conditional probabilities indicate unconditional ones, and vice-versa.  This might be less prevalent, but when conditioning on a certain feature, subsample of the corpus, etc.\ it is common to assume that the trends identified are indicative (or anti-indicative) of similar trends in the rest of the corpus.

    \item[Texas Sharpshooter Fallacy (post-hoc theorising)] Testing hypotheses on the data they were derived from, i.e.\ inspection then derivation then testing.  Also called `data dredging', this is a risk of large-scale community reuse of corpora.  A solution to this is to reuse sample \textsl{designs}, but not the actual data itself.

    \item[Availability Heuristic] The tendency for people to mentally overestimate the probability of features which they can immediately recall examples for\cite{tversky1973availability,schafer2014focused}.  This effect is likely to occur when examining corpora qualitatively, particularly if features are inspected in response to searches on particular features.
\end{itemizeTitle}

Many of these concerns are difficult to address without fully quantifying or automating analysis stages that are currently performed manually, something that is often a technical challenge.  Though reduction of their incidence is the goal of a corpus builder, there is often little that can be done directly prior to analysis.

The quantitative stages of corpus analysis are also not free from statistical challenge.  Many of these issues surround the use of models making flawed assumptions about variability\cite{kilgarriff2005language}, something that is a direct result of disagreements between sampling unit and analysis unit.

One approach to this is advocated by Wallis\cite{wallis2013vexed}, who proposes using a baseline linguistic feature as a control.  This serves to normalise the frequency against which one is comparing during statistical tests, but remains subject to issues of low power due to use of samples that contain too few independent sources.

Use of better-informed linguistic models in order to take account of non-orthogonal variation in linguistic features is also possible.  One method of doing this is to adjust for (or at least recognise qualitatively) the dispersion of features across samples\cite{gries2008dispersions,gries2010dispersions}.  Research is needed in order to best use these measures for qualitative interpretation (dispersion is already displayed in some popular tools\cite{anthony2011antconc}) and to adjust existing statistical procedures.




% \til{Add refs to those points made in CL'15's stats thingy}











% Now we shift onto the low-level focus of the thesis, web issues
\section{New Technologies} % "Opportunities and Challenges"
\label{sec:litreview:newtech}

\subsection{Web Corpora}
The rise in popularity of the web introduced a significant new source of text for linguists.  Whilst computerised text has been included in many conventional corpora (in the form of email and other direct communications), the ubiquity of the web offers a source of digital-format documents that now cover many subjects and genres.

Due to widespread and diverse use of the web, such corpora span many populations---some efforts focus on describing the web itself, and representing the population of documents online as its own population.  Others are able to focus on more general representation, or offer metadata sufficient to retrieve special-purpose corpora.  It is certainly the case that any modern corpus claiming to be representative of production or reception should include web data.

The first efforts to construct corpora from web data\cite{kilgarriff2001web} focused on the production of tools and resources to download, clean, and index web data at large enough scales to be used in general purpose corpora.  Many of these problems have now been solved to some degree, with the WaC movement producing a number of web-specific corpus tools:


\begin{itemizeTitle}
    
    \item[Crawlers] WaC-specific crawlers have been produced that are able to control their behaviour according to linguistic properties\cite{schafer2014focused}, and with the intent to provide an overview of proportions online.

    \item[Boilerplate removal tools] Tools such as JusTexT\cite{pomikalek2013justext} and BoilerPipe\td{cite} are able to remove the `boilerplate' of websites in order to leave just the content areas.

    \item[Genre classification schemes] Taxonomies have been produced that include the new genres found online, or those new forms of text such as blogs.  Some of this work has come from search engine designers, eager to improve their results\cite{genreclassification2004}, as well as from linguistics proper\cite{sharoff2007classifying,santini10genreintro}.

    \item[Retrieval tools] Tools such as BootCaT\cite{baroni2004bootcat} and search engines such as WebCorp\cite{renouf2003webcorp} allow access to increasingly large-scale web resources.

\end{itemizeTitle}

The ease with which users may access web data is of particular value: this has allowed for corpora of unprecedented scale such as the WaCky\cite{baroni2009wacky} and COW\cite{schafer2012building} corpora (which contain tens of billions of words), as well as for near-instant corpus building and replication.

Many WaC efforts focus on sampling the web itself, in order to represent users' experience thereof.  This approach leads to sample designs that focus on uniform coverage of top-level domains, or particular types of web resource, in a manner similar to that of search engine crawlers.

Performing this task in a robust manner is particularly challenging due to the lack of any central index for websites at this level.  As such, it is common to oversample and then select documents after-the-fact using scoring techniques\cite{schafer2013web,schafer2014focused}.

A contrasting approach is that of using existing indices to retrieve data, usually by using commercial search engines.  This is the approach taken by the BootCaT tool, which relies on repeatedly requesting URLs from the Microsoft Bing search engine conforming to a number of seed terms.

This could be seen either as an attempt to construct a corpus in the image of the seed (hence the ``bootstrapping'' in the name), or, under conditions of sufficiently general input, one to represent the web itself, using the search engine as a `lens' through which to view web documents.  This is a distinction drawn by the source of variation (the seed terms themselves, or the web's ability to return the desired data).

This performance was examined by Kilgariff and Grefenstette\cite{kilgarriff2003introduction}, stating w.r.t. representativeness:

\begin{quote}
The Web is not representative of anything else. But neither are other corpora, in any
well-understood sense.
\end{quote}

This distinction is something I will revisit in Chapters~\ref{sec:longitudinal} and \ref{sec:evaluation}.



\subsection{Intellectual Property}
The legality of releasing large volumes of text has always been an issue for corpus builders, however, much of the time this issue has been handled by an intermediary: the publisher, able to provide rights to many documents at once.  WaC methods access content by many publishers, often where rights are poorly-stated or belong to owners which cannot be contacted (especially in bulk).  This issue is complicated further by the international nature of the web, and the style of hyperlinking and content embedding.  

One approach to this is to limit access to end users, providing an interface which still permits some level of inquiry whilst restricting large-scale access to the whole volume of data.  This is the approach taken by BE06\cite{baker2009be06}, which is available through CQPWeb even though it contains data that is nominally copyright to other parties.

A second approach is to provide lists of URLs from which the original data may be retrieved.  This approach is less legally troublesome than that taken by BE06, yet also requires a significant amount of work on behalf of any replication.

Finally, corpora such as the Google Books corpus\cite{goldberg2013dataset} and the COW corpora\cite{schafer2012building} are offered in a jumbled form, designed to retain frequency information without the ability to recover full texts.  Both corpora mentioned here are particularly large, implying that they are unlikely to be used by researchers reading the text directly due to the large number of results for even very specific queries.


I consider that these approaches miss the point somewhat, as the ease of retrieval WaC brings make it possible to perform full scientific replication: retrieving new data conforming to the same sampling policy, rather than repeating the data verbatim.



\subsection{Sample Design}
The web, and its ease-of-access, also enables new sample designs.

The most obvious of these is the automation of monitor corpus maintenance --- this is analogous to the problem of maintaining search engine indices, and as such there is significant literature from that area focused on the scheduling revisits to pages and ensuring coverage across domains.  These techniques are now widely used in lexicography to keep dictionary resources up-to-date.

The availability of translated documents online also provides easy sources of data for constructing parallel corpora, a task that can be almost completely automated due to web page markup\cite{resnik2003web}.

The ability to automate retrieval also means that diachronic designs can be automated to some extent.  This is explored and developed further in Chapter~\ref{sec:longitudinal}.

 % parallel corpora
 % diachronic corpora





% Further to this, digitisation has led to a tighter, more explicit focus on intellectual property rights and re-use, including the use of DRM to block access in some instances.

\section{Documents of Digital Origin}
With the rise of the paperless office (ha!), even documents typically accessed in physical form are authored and stored digitally.  This has now extended to almost every form of textual information.

This has two main advantages: firstly there is greater coverage of conventional formats such as books.  Secondly there are entirely new opportunities to sample and meaningfully process things that have never been available before, like flyers, video with overlays, etc.

The realisation of increasing quantities of data as native-digital objects means that it is possible to take copies with relatively low costs, and without depriving the original owner of the resource for any great length of time.

The increased breadth of use also lends itself to mulit-modal corpus designs, effectively expanding the coverage of a corpus to better suit its population.  This is a technique already seen in speech corpora.

% mention OCR and word lens?


\subsubsection{Life-Logging}
Life-logging is an activity that is focused around gathering, organising, and using a continuous record of the data encountered in everyday life.  It has been developed with two main focuses, both of which may lend value to the process of corpus building and sampling:

\begin{itemize}
    \item Entertainment---Many people have, since the mid 1990s, broadcasted significant portions of their life online, something that has risen in popularity to the point of spawning consumerised applications for the purpose (Justin.tv, ushare).  Methods used focus on audio-visual broadcasting, as the output is matched closely in format to reality TV.
    \item Information categorisation and extraction for personal use---This has been the main focus of the academic community (and, in one notable case, DARPA), and has spawned many projects that focus on digesting and operationalising lifelogging data.  Typically such efforts are less focused on audio-visual data, since it is prohibitively difficult to process.
\end{itemize}


With the availability of powerful portable devices such as tablets and phones (and especially Google Glass), life-logging techniques that have conventionally been restricted to only a few individuals worldwide due to technical requirements or practical limitations are becoming increasingly viable as sources of information for many people.

These techniques offer an approach to sampling that, whilst explored by many other fields, is typically seen as expensive and involved.  One's own logged history may be a useful source of data for systems that interact using NLP techniques (in order to mimic one's dialect and idiomatic language use more closely), or the language proportions of social groups may be more accurately determined for scientific study.  The value of such personalisation is already proven in many contexts with more limited interaction methods, for example speech recognition (personalised phonetic models) and web search (Google and others' personalised results).  Further to the benefits of being able to gather real data more easily, life-logging allows us to peer further into social contexts with less disruption, yielding higher quality data.

Where language metadata is needed (rather than verbatim text), many life-logging technologies support discarding of any identifiable information on-the-fly---there are techniques for storing only irreversably scrambled audio such that the characteristics of speakers may still be identified
\cite{lee2006voice}, or devices with sufficient power can simply store summaries of the events they observe, discarding the data itself.
The decrease in ethical sensitivity associated with such measures further reduces the boundaries to wider sampling of a population, something that may be used to improve and adapt existing languages resources.


% \til{By analogy, Google maps obscuring people's faces using face recognition tech}

Such life-logs, even heavily anonymised, offer new opportunities for balancing corpus strata, as well as providing a perspective on use that is more richly annotated with contextual metadata.

% \til{stress further the ability to use stuff as auxiliary data for balancing, rather than comprising, a corpus.  link forward to personal corpus stuff.}
% \til{ Perhaps a lit review belongs here, but the structure falls too deeply to do it in a structured, logical way...}




\subsection{New Sampling Opportunities}
Miniaturisation of technology, and increases in computing power, gradually unveils new opportunities to sample data.  At one level, this may make possible voice recognition and transcription of multiple subjects, or analysis of data in-place, without transcription.  At another, computationally-expensive techniques such as MCMC become increasingly applicable to samples large enough to be used for linguistic purposes.


\subsubsection{Access to Technology}
The ubiquity of technology makes accessing a diverse popuilation of languiage speakers with little concern for geographic limitations.  Submissions of textual data may quickly be acquired via the intetnet, and populations of people otherwise unrepresented in corpora may be sampled this way.


This has also had other effects, such as the tendency for a single conversation online to include speakers from many countries, cultures and demographics (often without even being aware of that fact).  Also, certain subcultures use various specialised language forms online that can squirrel off into their own little thing with little reference back to 'everyday' forms.

One effect of this has been the ability to recruit study participants from across the world at relatively little cost through the use of crowdsourcing platforms such as Amazon AMT\cite{ipeirotis2010analyzing} or Crowdflower\cite{Finin:2010:ANE:1866696.1866709}.  This widespread ability brings with it a significant number of analytical challenges, many of which are still being worked upon, and many of which, such as the problem of managing inter-annotator agreement with many participants, apply nicely to the problem of sampling quality corpus data in a distributed manner.

  


\subsubsection{Indexing and Access}
The improved power and utilisation of large-scale computing machinery opens up possibilities for more complex examination and extraction of data from existing lists.  A prime example of this is the pooling of data in and around search engines such as Google, which have become a source all of its own for many researchers.

Once data is acquired, data warehousing techniques open possibilities for examination using many more covariates than has previously been possible, allowing very complex research questions to construct meaningful subcorpora with relatively little effort or time overhead.

This bulk analysis of heterogenous data is regarded as a growth area in many circles, though seems to have been under-utilised in academia due to its low quality compared to traditional scientific samples, and the low replicability that implies.  Whether 'big enough' big data is ever useful in a scientific context remains to be seen, however, this is arguably something WaC already embraces.


% ccil{mention biber's "only the weird bits are interesting" point, find some big data introductory refs}




