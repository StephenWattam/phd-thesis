
\section{Introduction}
Taking the opportunities and challenges outlined in previous sections, it is possible to assess sampling strategy online in its own context, making use of the speed and ease-of-access that the web yields whilst mitigating the practical issues outlined in [chapter 3].

In part due to difficulties evaluating corpora (covered in more detail in chapter 6), many methods for web corpus construction are based on, and compared against, current conventional corpora.  I start with a review of these, identifying places where these methods may be augmented or improved upon.

`Proportionality' discusses possible solutions to the problem of selecting stratum sizes within a corpus where no authoritative central index or information about consumption exists.  For this, I turn to life-logging methods.

The final section describes a method for describing corpora according to a series of extracted variables.  This is intended to provide an alternative to description based on URLs, addressing the copyright and dissemination concerns outlined in the previous chapter and providing a more scientifically robust method of repeating studies.



\section{Methods for Improved Web Sampling}
Based upon information from [ref: chap:attrition], and the biases/discussion identified at the end of the lit review, derive methods to fix issues identified.  Separate them into the general-purpose/specialist framework again, and their theoretical justification.  Compare current linguistic methods to the sampling methods in soc. science.

\subsection{Current Methods}
Cover tools and how they go about sampling.  Typically this is a spider/search thing, though things like webcorp complicate that categorisation.


% I'm not convinced this is the best structure
\subsection{Top down (linguistically justified)}
\begin{itemize}
    \item Personal corpora,
    \item Subject-specific sampling methods (though these don't map well to the web).
\end{itemize}
\subsection{Bottom up (statistically justified)}
\begin{itemize}
	\item Social Science's methods for balancing data where population, distribution are hard to determine or where data acquisition is hard,
	\item Attempts made by others to compensate in linguistic studies.
\end{itemize}



\til{chapter 4 cut from here}




\section{Corpus Description Language}
Cover what CDL aims to do, and what problems this should solve (dissemination, data dredging, comparison and proportionality).


\section{Describing a Sampling Policy}
This will focus on two main aspects of description, but both will lean towards being algorithmically described for use in software:
\begin{itemize}
	\item Non-URI methods for describing documents online (i.e. methods including linguistic descriptions);
	\item Methods for describing, in general, a sample to an algorithm (presumably these exist already, but I'm unaware of them);
	\item Methods of evaluation ("distance between samples in terms of samples").
\end{itemize}




\subsection{Design}
Cover use-cases, do some fun diagrams.


\subsection{Implementation}
Detail the final form of the language and the modules therein.  Relate to Habeas in a foreshadowy way.



